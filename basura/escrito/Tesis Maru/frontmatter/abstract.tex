


Los jets son un ingrediente fundamental en muchos de los estudios que se llevan a cabo en ATLAS en el LHC ya que son los objetos de estado final predominantes en las colisiones protón-protón. La calibración oficial de jets en ATLAS corresponde a jets reconstruidos con un radio de tamaño $R=$0.4, sin embargo, poder calibrar jets reconstruidos con un algoritmo de distinto tamaño es de gran utilidad en diversos análisis físicos. El tiempo y el esfuerzo que se requiere para derivar una calibración completa hace que no sea viable derivar tantas calibraciones como radios sean de interés. En esta tesis se estudia un nuevo método de inter-calibración de jets llamado calibración R-scan. El objetivo es derivar calibraciones para colecciones de jets de algún tamaño $R$ en datos utilizando como objetos de referencia los jets reconstruidos de manera estándar con $R=$0.4. En particular, se implementa el método de direct-matching el cual propone identificar geométricamente el jet que se quiere calibrar con el jet de referencia. Se estudiaron eventos de $Z(\rightarrow\mu\mu)+jets$ en datos de colisiones $pp$ a $\sqrt{s}=$13TeV recolectados por el detector ATLAS durante el año 2016 y en simulaciones de Monte Carlo (MC). Se obtuvieron dos calibraciones, una para $R=$0.2 y para $R=$0.6, junto con sus incertezas sistemáticas en función de la pseudo-rapidez y el momento transverso del jet a calibrar. En la región de $|\eta|<$0.8, la calibración para $R=$0.2 es válida en el rango 15GeV$<p_t<$290GeV, mientras que la calibración para $R=$0.6 lo es en el rango 25GeV$<p_t<$300GeV. Para ambas se estima una incerteza relativa total del 5$\%$ a bajo momento y del $<2\%$ a alto momento. Se realiza también un estudio de validación de las calibraciones derivadas aplicándolas en datos y buscando diferencias residuales entre datos y MC. Por último, se hace una comparación de la calibración derivada para $R=$0.6 en esta tesis, con una calibración R-scan derivada en eventos de muchos jets, observándose un acuerdo entre ambas al considerar incertezas asociadas en el rango de momento en el que se solapan. 

%\newpage
%\blankpage
