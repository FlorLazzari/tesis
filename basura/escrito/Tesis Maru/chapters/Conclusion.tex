\chapter{Conclusiones}\label{Conclus}

El proceso estándar de calibración de jets es uno que ocupa gran cantidad de tiempo y esfuerzo humano dentro del experimento ATLAS. El hecho de que la última calibración oficial\cite{JESpaper} de ATLAS derivada para jets reconstruidos con $R=$0.4 haya demorado alrededor de dos años en publicarse es motivo suficiente para afirmar que no es factible producir una calibración para cada radio de interés. 

Esto representa un obstáculo para los diferentes estudios que allí se llevan a cabo. Por ejemplo, distintas mediciones de precisión de SM reclaman una calibración para jets de radio 0.6 para reducir el impacto de las incertezas causadas por radiación que cae fuera del cono del jet cuando este es reconstruido con un radio 0.4. Otro ejemplo que se puede mencionar son los estudios de física boosteada donde se requiere una calibración para jets de radio 0.2 para poder analizar la subestructura de jets que engloban a los decaimientos de una partícula boosteada. Este tipo de física está presente en muchas búsquedas \textit{beyond-SM}.  

Una solución a la creciente demanda de las mismas es el método de R-scan. Este es un método sencillo, de tipo in-situ, que propone calibrar jets de tamaño $R$ respecto de jets reconstruidos y calibrados de manera estándar. 

En el marco de esta tesis, y utilizando los datos de colisiones $pp$ a $\sqrt{s}=$13TeV recolectados por el detector ATLAS en 2016, se derivaron dos calibraciones Rscan, una para $R=$0.2 y otra para $R=$0.6, junto con sus correspondientes incertezas. Además, se realizó una validación de las mismas al aplicárselas a los Rscan jets en datos y concluir que no se observan diferencias residuales (con una tolerancia del $1\%$) entre datos y MC. Finalmente, se realizó una comparación en la que la calibración para $R=$0.6 derivada en eventos de $Z+jets$ en este trabajo resultó ser consistente con otra calibración Rscan derivada en eventos de muchos jets en el rango de momento transverso en el que se solapan.     




