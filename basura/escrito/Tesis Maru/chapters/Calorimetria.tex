
\section{Calorimetría}

En detectores como ATLAS en el LHC las partículas se detectan reconstruyendo su traza y midiendo su energía. En esta sección se introduce el concepto de calorimetría desde el punto de vista de la interacción de las partículas con el material del detector, y un detalle de cómo se reconstruyen las trazas y el lay-out del experimento ATLAS se describe para el capítulo \ref{TheExperiment}.

En física de partículas, un calorímetro es un tipo de detector que mide la energía de una partícula frenándola por completo y traduciendo esta energía en una señal eléctrica. Como consecuencia, la partícula ya no se encuentra disponible para más mediciones. Si la energía de la partícula se encuentra bien por encima del umbral de scattering inelástico entre dicha partícula y el material del detector, el proceso de pérdida de energía se da con una cascada de partículas de menor energía, en número acorde con la energía incidente. Las partículas cargadas en la lluvia eventualmente pierden su energía a través de procesos más elementales, principalmente, ionización y excitaciones a nivel atómico. La señal calorimétrica corresponde a la suma de estas pérdidas elementales. Sólo interacciones de naturaleza electromagnética o fuerte contribuyen a las señales calorimétricas, partículas que sólo pueden interactuar débilmente escapan de la detección calorimétrica.

En simulaciones de MC, para reproducir las señales calorimétricas medidas en ATLAS, los eventos generados a ``truth level'' son propagados a través de una simulación completa del detector basada en GEANT4 \cite{AtlasSimulation}. A esta altura, las simulaciones de MC y los datos recolectados del detector están en pie de igualdad, de modo que en ambos casos se pueen utilzar las mismas herramientas de reconstrucción de objetos.

\subsection{Interacción de las particulas con la materia}

La interacción de las partículas con el material del detector puede darse de diferentes maneras\cite{Instrumentation}:\\
\indent \emph{Ionización y excitación atómica}. Una partícula cargada que atraviesa un átomo puede interactuar via la fuerza de Coulomb con los electrones atómicos.
Si la transferencia de energía es pequeña (una interacción distante) el electrón atómico pasará a un estado excitado; mientras que si la transferencia de energía es lo suficientemente grande para sueperar la energía de ligadura, el átomo se ionizará y el electrón atómico será liberado. Los fotones resultantes de la des-excitación de los átomos y los electrones e iones producto de la ionización son capaces de generar señales medibles.\\ 
\indent \emph{Scattering múltiple, bremsstrahlung y producción de pares}. La interacción Coulombiana de una partícula con los núcleos atómicos del material del detector produce una desviación de la trayectoria de la partícula, esto recibe el nombre de ``scattering múltiple''. Esta deflección induce una aceleración, y, por lo tanto, la emisión de radiación electromagnética, efecto denominado Bremsstrahlung. Además, un fotón de alta energía tiene una probabilidad de producir un par electrón-positrón en la vecindad de los núcleos. La distancia media que viaja un fotón de alta energía en el material antes de convertirse en un par electrón-positrón está dada aroximadamente por la longitud de radiación (distancia recorrida para la cual la energía del electrón cayó a $1/e$ de la energía incial). Estos dos procesos dados alternadamente resultan en una cascada electromagnética de cada vez más electrones y positrones de menor energía, hasta que se ``detienen'' en el material cuando su energía cae por debajo de una energía crítica a partir de la cual disipan su energía por ionización y excitación.\\
\indent \emph{Radiación Cerenkov}. Las partículas cargadas que atraviesan un material a velocidades mayores que la velocidad de la luz en dicho material producen una onda de choque electromagnética que se evidencia como radiación electromagnética en el rango visible y ultravioleta.\\
\indent \emph{Radiación de transición}. La radiación de transición se emite cuando una partícula cargada atraviesa la interfaz entre dos materiales de diferente permitividad. La probabilidad de emisión es proporcional al factor de Lorentz $\gamma$ de la partícula, y sólo resulta apreciable para partículas ultra relativistas, con lo cual es útil para distinguir electrones de hadrones. 

\subsection{Lluvias electromagnéticas}
 Como ya se mencionó, el proceso de creación de pares (dominante para fotones con energía por encima de $2\,m_ec^2$) alternado con Bremsstrahlung (dominante para electrones y positrones por encima de una energía crítica $\approx 550 MeV/Z$) lleva a un electrón o fotón a producir una lluvia de electrones o positrones, con energías cada vez menores hasta que se detienen en el material debido a pérdida por ionización. La cantidad total de ionización producida da una medida de la energía de la partícula.
 
 \subsection{Lluvias hadrónicas}\label{Lluvias}
 Existe el efecto análogo al bremsstrahlung electromagnético para interacciones  hadrónicas. Cuando un hadrón de alta energía interactúa con el material del detector, tipicamente choca contra un núcleo atómico y produce la fragmentación del mismo y hadrones secundarios que luego pueden interactuar con otros núcleos o bien decaer en otros hadrones.
 
 La lluvia hadrónica continuará hasta que los hadrones no tengan suficente energía para romper más núcleos. En este punto la pérdida de energía se dará principalmente por ionización o absorción en algún proceso nuclear\cite{DetectorBook6}. La escala de longitud en este caso es la longitud de interacción hadrónica y es significativamente mayor que la longitud de radiación mencionada para el caso electromagnético (por ejemplo, para el caso del hierro, es diez veces mayor). Esto quiere decir que las lluvias hadrónicas deben avanzar sobre más material que una lluvia EM para depositar toda su energía. Es por esto que los calorímetros hadrónicos suelen ser de materiales más densos y estar ubicados más allá de los EM \cite{Instrumentation}.
 
 En una lluvia hadrónica, la energía se transfiere al medio a través de interacciones electromagnéticas (ionización, $\pi^0 \rightarrow \gamma\gamma$, etc.) y hadrónicas (fisión, recombinación en hadrones, scattering elástico de neutrones, etc.). En la componente hadrónica, la energía utilizada en ligaduras nucleares, excitaciones nucleares, y en reducir la velocidad de los neutrones no puede ser detectada por el calorímetro. De la misma manera, la producción de neutrinos y muones que escapan del detector da por resultado energía que tampoco puede ser detectada. Es por este motivo que generalmente la eficiencia al medir energía depositada via interacciones electromagnéticas es mayor que en el caso hadrónico. 
 