\chapter*{Resumen}  % No aparece en el indice 

\addcontentsline{toc}{chapter}{Resumen} %% aparece en el indice pero no como un cap.

\markboth{}{Resumen}

    

La medición de actividad y variabilidad estelar se realiza, casi con exclusividad, utilizando como indicador el flujo de las líneas H y K del Ca \textsc{ii} del extremo azul del espectro visible. Los estudios observacionales sistemáticos realizados hasta el momento, dedicados principalmente a estrellas en el rango F a K tempranas, ha detectado ciclos estelares similares al ciclo solar. Dado los campos magnéticos que causan la actividad estelar son generados por la interacción entre la convección y la rotación estelar (dínamo estelar), las estrellas tardías en sistemas binarios presentan altos niveles de actividad, debido a su alta tasa de rotación ya que se encuentra forzada por su compañera a través de las fuerzas tidales. Uno de los sistemas binarios más activos son los sistemas de tipo \textit{RS Canum Venaticorum}. El estudio de periodicidad de la estrella primaria de estos sistemas resulta particularmente interesante, ya que brinda información directa sobre la relación entre rotación y variabilidad, y en última instancia, sobre los mecanismos responsables del dínamo estelar. En el siguiente trabajo se focalizó el estudio de actividad magnética de largo plazo de sistemas de tipo RS CVn a partir de observaciones espectroscópicas propias obtenidas en CASLEO y fotométricas públicas provenientes de ASAS.
    
    
    

