\chapter{Campos de Jacobi y Ecuación de Raychaudhuri}\label{capitulo Jacobi}

En este capítulo terminamos de presentar los resultados necesarios para poder enunciar y demostrar los Teoremas de Singularidad en el siguiente capítulo. Se trata del capítulo más extenso de la tesis por lo que se trató de dejar de lado ciertas demostraciones (las cuales el lector puede consultar en la bibliografía) para volver amena su lectura. Se siguieron como referencia a \citep{Carroll,1984ucp..book.....W,1975lsss.book.....H,Penrose}





%%%%%%%%%%%%%%%%%%%%%%%%  Ecuación de desviación geodésica - Ecuación de Jacobi

\section{Ecuación de desviación geodésica - Ecuación de Jacobi}\label{ec. desviacion geodesica}

Como se ha dicho en la sección \ref{curvatura}, el tensor de Riemann habla sobre el transporte paralelo. En un espacio plano (como se dijo en dicha sección) el transporte paralelo no depende del camino: líneas paralelas inicialmente, permanecen paralelas. Esto, sin embargo, no es cierto para espacios curvos (tales como una esfera) y la noción de paralelismo no puede extenderse de una forma natural a partir de espacios planos. Sin embargo, podemos considerar curvas geodésicas que inicialmente parecieran paralelas, y ver su comportamiento cuando se las traslada a través de geodésicas. Para ello consideremos una familia de geodésicas $\gamma_s(t)$ (uniparamétricas), donde para cada $s\in \mathbb{R}$, $\gamma_s$ es una geodésica parametrizada con parámetro afín $t$. La colección de dichas curvas define una superficie suave 2-dimensional y podemos tomar como coordenadas en dicha superficie a $s$ y $t$. De esta forma surgen naturalmente dos vectores: el vector tangente a la familia de geodésicas $T^a=(\partial/\partial t)^a$ y el vector de desplazamiento $S^a=(\partial/\partial s)^a$ que representa el desplazamiento infinitesimal de geodésicas cercanas (Figura \ref{fig_geodesicas}).

\begin{figure}[H]
    \centering
    \includegraphics[width=0.5\textwidth]{./img/Cap4/geodesicas.png}
    \caption{Conjunto de geodésicas $\gamma_s(t)$ con vector tangente $T^a$ y vector de desviación $S^a$.}
    \label{fig_geodesicas}
\end{figure}
    


 Definidos así, $S$ y $T$ cumplen que conmutan ya que son vectores coordenados y, por lo tanto, se satisface

\begin{equation}\label{conmutador}
     T^b\nabla_bS^a=S^b\nabla_bT^a 
\end{equation}
 

Se define $v^a=(\nabla_TS)^a=T^b\nabla_bS^a$ como la tasa de cambio a lo largo de una geodésica del desplazamiento a una geodésica infinetesimalmente cercana. Es decir, podemos interpetrar a $v^a$ como una velocidad relativa entre geodésicas infinetesimalmente cercas. A su vez, podemos definir $a^a=(\nabla_Tv)^a=T^b\nabla_bv^a$ e interpretarla como la aceleración relativa de geodésicas cercanas. A continuación, reescribiremos la aceleración relativa y la relacionaremos con el tensor de Riemann, llegando así a la ecuación conocida como \textit{ecuación de desviación geodésica}:

$$
a^a=T^b\nabla_bv^a=T^b\nabla_b(T^c\nabla_cS^a)=T^b\nabla_b(S^c\nabla_cT^a)
$$

en donde se ha usado que $S$ y $T$ conmutan (\ref{conmutador}). Aplicando la regla de Leibnitz se obtiene

$$
a^a=(T^b\nabla_bS^c)(\nabla_cT^a)+(T^bS^c)(\nabla_b\nabla_cT^a)
$$

De la definición del tensor de Riemann (\ref{definicion Riemann}) se deduce que, como $[S,T]=0$, entonces para algún vector $U$, $\tensor{R}{^a_{dbc}}U^d=(\nabla_b\nabla_c-\nabla_c\nabla_b)U^a$. Reemplazando esto en la ecuación anterior se obtiene que

$$
a^a=(T^b\nabla_bS^c)(\nabla_cT^a)+S^c\nabla_c(T^b\nabla_bT^a)-(S^c\nabla_cT^b)(\nabla_bT^a)+\tensor{R}{^a_{dbc}}T^dT^bS^c
$$

Notemos que el primer y el tercer término se anulan ya que $S$ y $T$ conmutan. Por otro lado, el segundo término se anula también ya que $\nabla_TT=0$ por tratarse de geodésicas. De esta forma resulta

$$
a^a=\tensor{R}{^a_{dbc}}T^dT^bS^c
$$

o, usando la notación $\frac{DS}{Dt}=T^c\nabla_cS$ es lo mismo que

\begin{equation}\label{Jacobi}
     \frac{D^2S^a}{Dt^2}=\tensor{R}{^a_{dbc}}T^dT^bS^c
\end{equation}

que es la llamada \textbf{ecuación de Jacobi} o también conocida como \textbf{ecuación de desviación geodésica}. Cabe notar que, usando las propiedades del tensor del Riemann expuestas en la sección \ref{curvatura}, la ecuación de Jacobi se puede reescribir como

$$
\frac{D^2S^a}{Dt^2}=-\tensor{R}{_{cbd}^a}S^bT^cT^d
$$

que es como muchas veces se presenta en la bibliografía. La ecuación de Jacobi (o de desviación geodésica) relaciona la aceleración relativa con la curvatura: $a^a=0$ sii $\tensor{R}{^a_{dbc}}=0$, por lo tanto algunas geodésicas se acercarán o se alejarán unas de otras sii $\tensor{R}{^a_{dbc}}\neq 0$. 

A continuación se definen dos conceptos de suma importancia:

\begin{definition}
Sea $\gamma$ una geodésica con vector tangente $T^a$; $X^a$ se dice que es un \textbf{campo de Jacobi} en $\gamma$ si cumple la ecuación de Jacobi.
\end{definition}

\begin{definition}
Un par de puntos $p,q\in \gamma$ se dicen \textbf{conjugados} si existe un campo de Jacobi en $\gamma$ no idénticamente nulo pero que se anula en $p$ y $q$. 
\end{definition}

Esta última definición será de importancia para el uso de los Teoremas de Singularidades y será ampliada en la sección \ref{pts conjugados}.




%%%%%%%%%%%%%%%%%%%%%%%%  Espacio de Curvas Causales $C(p,q)$

\section{Espacio de curvas causales $C(p,q)$}\label{seccion Cpq}


En general, en un espacio-tiempo arbitrario, no es cierto que dados dos puntos exista una geodésica que los una \citep{Beem}. Sin embargo, en esta sección definiremos el espacio de curvas causales que conectan dos puntos en un espacio-tiempo globalmente hiperbólico, el cual ayudará para poder dar ciertos resultados posteriores sobre la existencia de curvas (geodésicas) que maximizan la longitud entre dos puntos. Para ello, sea un espacio-tiempo fuertemente causal $(M,g_{ab})$ y sean $p,q\in M$. Se define $C(p,q)$ como el conjunto de curvas continuas causales futuro directo que van de $p$ a $q$, en donde las curvas que difieren por una reparametrización son consideradas como una misma curva. Podemos dotar a este espacio de un topología de la siguiente manera: dado un abierto $U\subset M$, se define $O(U)\subset C(p,q)$ como

\[ O(U)=\{\lambda\in C(p,q) \mid \lambda \subset U\} \]

Esta definición equivale a decir que $O(U)$ consiste en todas las curvas causales que conectan $p$ y $q$, y que están completamente contenidas en $U$. A su vez, $O\subset C(p,q)$ es abierto si se puede escribir como unión de elementos de la forma $O(U)$. 

Se puede probar que el espacio topológico definido previamente es de Hausdorff y, cuando no existen curvas causales cerradas, también es segundo contable \citep{1970JMP....11..437G}. Sin embargo, un resultado más importante se enuncia a continuación:

\begin{theorem}\label{C(p,q) compacto}
Sea $(M,g_{ab})$ un espacio-tiempo globalmente hiperbólico y sean $p, q\in M$, entonces $C(p,q)$ es compacto.
\end{theorem}
\begin{proof}
Como la topología en $C(p,q)$ es segundo contable, por el Teorema \ref{B-W} basta con probar que cada sucesión de curvas $\{\lambda_n\}$ posee un punto de acumulación (i.e: una curva límite $\lambda$) en $C(p,q)$. Sea $\Sigma$ una hipersuperficie de Cauchy de $(M,g_{ab})$; consideremos el caso en el que $p,q\in D^-(\Sigma)$ y sea $\{\lambda_n\}$ una sucesión de curvas en $C(p,q)$ tal como muestra la figura \ref{demoC(p,q)}. Si removemos el punto $q$ del espacio, entonces $\{\lambda_n\}$ se convierte en una sucesión de curvas causales intextensibles al futuro comenzando en $p$. Por el Lema \ref{existencia curva limite}, existe una curva causal inextensible al futuro, $\lambda$, que comienza en $p$ y es curva límite de $\{\lambda_n\}$. Como ninguna de las curvas $\lambda_n$ pasan por $I^+(\Sigma)$, tampoco lo hace $\lambda$. Si, ahora, recuperamos el punto $q$ removido del espacio, entonces puede suceder que $\lambda$ siga siendo inextensible o bien puede suceder que $q$ sea punto final de $\lambda$. Sin embargo, lo primero no puede suceder ya que $\lambda$ no pasa por $I^+(\Sigma)$. Por lo tanto, $\lambda$ (con su punto final $q$ incluído) será la curva límite deseada. Un razonamiento análogo se sigue para $p,q\in D^+(\Sigma)$. El caso no trivial remanente es pues en el que $p\in D^-(\Sigma)$, $q\in I^+(\Sigma)$. Dada una sucesión $\{\lambda_n\}$ en $C(p,q)$, un razonamiento como el previo (aplicado en $(M-q)$) permite obtener una curva límite futuro directo $\lambda$, que comienza en $p$ y entra en $I^+(\Sigma)$. Sea $r\in \lambda \cap I^+(\Sigma)$, y sea $\{\lambda'_n\}$ una subsuceción de $\{\lambda_n\}$ tal que cada punto en el segmento de $\lambda$ entre $p$ y $r$, es un punto de convergencia de esta subsucesión. Repitiendo el razonamiento para la subsucesión $\{\lambda'_n\}$ que comienza en $q$,  ahora en $(M-p)$ (i.e: estoy considerando ahora curvas causales inextensibles al pasado), obtenemos una curva límite $\lambda'$ que entra en $I^-(\Sigma)$ y pasa a través de $r$, ya que $r$ es un punto de convergencia de $\{\lambda'_n\}$ y si $\lambda'$ no se extendiese hasta $r$ entonces debería permanecer en $I^+(r)\subset I^+(\Sigma)$. Por lo tanto, uniendo el segmento de $\lambda'$ de $r$ a $q$ con el segmento de $\lambda$ de $p$ a $r$ se obtiene la curva límite deseada.
\end{proof}

\begin{figure}[h]
    \centering
    \includegraphics[width=0.5\textwidth]{./img/Cap4/demo421.png}
    \caption{Sucesión $\{\lambda_n\}$ de curvas causales que unen $p$ y $q$ en el caso en que $p,q\in D^-(\Sigma)$, usada para la demostración del Teorema \ref{C(p,q) compacto}.}
    \label{demoC(p,q)}
\end{figure}


Para finalizar la sección enunciaremos un resultado (sin demostración, ver \citep{1975lsss.book.....H} Proposición 6.6.2 ) de interés:


\begin{proposition}
Sea $N$ un espacio-tiempo fuertemente causal. Entonces $N$ es globalmente hiperbólico sii $C(p,q)$ es compacto para todo $p,q \in N$. 
\end{proposition}




%%%%%%%%%%%%%%%%%%%%%%%%  Condiciones de Energía

    
\section{Condiciones de energía}



A la hora de tratar de resolver las ecuaciones de Einstein, uno de los inconvenientes es el hecho de saber qué fuente se encuentra presente en la misma. Para limitar la arbitrariedad que pueda llegar a tener el tensor de energía-momento, se imponen ciertas \textbf{condiciones de energía} sobre el mismo.

Las condiciones de energía son restricciones - invariante frente a cambio de coordenadas - que se le hace al tensor de energía-momento. Para ello, debemos construir escalares a partir de $T_{ab}$, formados generalmente a partir de contracciones con vector temporales o nulos. A continuación veremos un ejemplo en particular y, posteriormente, enunciaremos otras condiciones de energía presentando los resultados pertinentes.

La \textit{condición de energía débil} (WEC por sus siglas en inglés) establece que $T_{ab}t^at^b\geq 0$ para todo vector temporal $t^a$. Para fijar ideas, consideremos el caso de un fluido perfecto cuyo tensor de energía-momento viene dado por

\[ T_{ab}=(\rho+p)U_aU_b+pg_{ab}\]


donde $U^a$ es la velocidad del fluido. Como la presión es isótropa, entonces $T_{ab}t^at^b$ será no negativo para todo vector temporal $t^a$ si $T_{ab}U^aU^b\geq0$ y si $T_{ab}k^ak^b\geq0$ para algún vector nulo $k^a$. Por lo tanto resulta

\begin{align*}
    T_{ab}U^aU^b=\rho&   & T_{ab}k^ak^b=(\rho+p)(U_ak^a)^2
\end{align*}

Se deduce, pues, que la WEC implica que $\rho\geq0$ y $\rho+p\geq0$. Estas condiciones se traducen en decir que la densidad de energía sea no negativa y que la presión no sea muy grande comparada con la densidad de energía.

Haciendo razonamientos análogos al anterior se pueden obtener diversas condiciones de energía, las cuales enunciaremos las más conocidas a continuación:


\begin{itemize}
    \item \textbf{Condición de energía débil (WEC)}: Tal como se mostró previamente, esta condición establece que $T_{ab}t^at^b\geq0$ para todo vector temporal $t^a$. Esto es lo mismo que decir que $\rho\geq0$ y $\rho+p\geq0$.
    \item \textbf{Condición de energía nula (NEC)}: Establece que $T_{ab}k^ak^b\geq0$ para todo vector nulo $k^a$. Es un caso especial de la WEC, donde se reemplazan los vectores temporales por vectores nulos. Equivalentemente, $\rho+p\geq0$. En este caso, la densidad de energía puede ser negativa siempre y cuando haya una presión positiva que la compense.
    \item \textbf{Condición de energía dominante (DEC)}: La WEC está contenida aquí ($T_{ab}t^at^b\geq0$ para todo vector temporal $t^a$) pero también se agrega una condición extra: $T^{ab}t_a$ es un vector causal. En el caso de un fluido perfecto, esto se traduce en decir que $\rho\geq|p|$: la densidad de energía debe ser no negativa y mayor o igual que la presión (en módulo).
    \item \textbf{Condición de energía dominante nula (NDEC)}: Es la misma condición que la DEC pero para el caso de vectores nulos: $T_{ab}k^ak^b\geq0$ para cualquier vector nulo $k^a$, y $T^{ab}k_a$ es un vector causal. Las densidades de energía y la presión son iguales a la DEC, pero con la excepción de que en este caso sí se pueden tener densidades de energía negativas, siempre y cuando se satisfaga $p=-\rho$.
    \item \textbf{Condición de energía fuerte (SEC)}: Establece que $T_{ab}t^at^b\geq\frac{1}{2}\tensor{T}{^c_c}t^dt_d$ para todo vector temporal $t^a$. Equivalentemente, $\rho+p\geq0$ y $\rho+3p\geq0$. Notemos que la SEC implica la NEC pero excluyendo valores de presiones negativas demasiado grandes. A su vez, es la SEC la condición que implica que la fuerza gravitatoria sea atractiva \citep{Carroll}.
\end{itemize}

Cabe destacar que las condiciones de energía no son teoremas de conservaciones de energía, sino que previenen fuentes ``no físicas" de aparecer en la teoría tales como alguna donde la energía se propague a mayor velocidad que la de la luz, o regiones del espacio vacías donde espontáneamente aparecen energías positivas y negativas. 

Los campos considerados a lo largo de la tesis son campos clásicos, pero cabe destacar que si se tratara de campos cuánticos, es posible que se violen alguna/s de las condiciones de energías previamente dichas. Sin embargo, es posible dar condiciones - que involucran integrales sobre regiones del espacio-tiempo - que sí cumplan los campos cuánticos. Una discusión más en detalle sobre el tema se puede encontrar en \citep{2011CQGra..28l5009F}.






%%%%%%%%%%%%%%%%%%%%%%%%  Congruencia de Geodésicas - Ecuación de Raychaudhuri

    
\section{Congruencia de geodésicas - Ecuación de Raychaudhuri}\label{seccion eq Raychaudhuri}



En la sección \ref{ec. desviacion geodesica} arribamos a la ecuación de desviación geodésica, la cual relaciona la evolución del vector de desplazamiento con geodésicas cercanas. Una visión más completa del mismo se podría pensar si en vez de considerar una familia de geodésicas uniparamétricas, se considera una \textit{congruencia} de geodésicas. Una \textbf{congruencia} es un conjunto de curvas en una región abierta del espacio-tiempo tal que cada punto en esa región es cruzado por una única curva. Formalmente, si $O\subset M$ es un abierto, una congruencia en $O$ es una familia de curvas tal que a través de cada $p\in O$, pasa exactamente una curva de esta familia. Una idea intuitiva es que podemos pensar a una congruencia como un ``manojo" de curvas. Los vectores tangentes a esta congruencia generan un campo vectorial en $O$ (y es posible probar que vale la inversa también \citep{1984ucp..book.....W}), y la congruencia se dice suave si el campo vectorial correspondiente lo es. A su vez, si hay geodésicas que se cruzan en la congruencia, entonces la misma necesariamente llega a un punto final precisamente donde se cruzan las geodésicas.

A continuación daremos ciertas definiciones para luego arribar a resultados de sumo interés para el estudio de los Teoremas de Singularidades. Haremos la distinción entre congruencias de geodésicas temporales y geodésicas nulas, llegando así a resultados análogos que se generalizan para geodésicas causales.



%%%%%%%%%   Geodésicas temporales

    
\subsection{Geodésicas temporales}\label{geodesicas temporales}

Consideremos una congruencia suave de geodésicas temporales. Sin pérdida de generalidad, parametrizaremos a las geodésicas con tiempo propio $\tau$ de manera que el campo vectorial $V^a$ de vectores tangentes a las geodésicas quede normalizado: $g_{ab}V^aV^b=-1$.



\begin{definition}
Se define la \textit{métrica espacial} como $h_{ab}=g_{ab}+V_aV_b$
\end{definition}

Algunas propiedades que satisface la métrica espacial son:

\begin{enumerate}[1)]
    \item $\tensor{h}{^a_a}=\tensor{g}{^a_a}+V^aV_a=3$
    \item $h_{ab}V^a=g_{ab}V^a+V^aV_aV_b=0$
    \item $h_{ab}\tensor{h}{^a_c}=(g_{ab}+V_aV_b)(\tensor{g}{^a_c}+V^aV_c)=g_{bc}+V_bV_c=h_{bc}$
\end{enumerate}
 
 
De aquí se deduce que por lo tanto podemos interpretar a $\tensor{h}{^a_b}=g^{ac}h_{cb}$ como un operador de proyección al subespacio (del espacio tangente) perpendicular a $V^a$. En términos de la métrica espacial $h_{ab}$ y de los vectores temporales $V^a$, se definen los siguientes tensores:

\begin{itemize}
    \item \textbf{Tensor de vorticidad}: $\omega_{ab}=V_{[c;d]}\tensor{h}{^c_a}\tensor{h}{^d_b}$
    \item \textbf{Tensor de expansión}: $\theta_{ab}=V_{(c;d)}\tensor{h}{^c_a}\tensor{h}{^d_b}$
    \item \textbf{Escalar de expansión}: $\theta=\theta_{ab}g^{ab}$
    \item \textbf{Tensor de corte}: $\sigma_{ab}=\theta_{ab}-\frac{1}{3}\theta h_{ab}$
\end{itemize}


Si nos remitimos a la sección \ref{ec. desviacion geodesica}, de la ecuación de desviación geodésica definiremos $B_{ab}=\nabla_bV_a$. Dicho tensor es espacial ($B_{ab}V^a=B_{ab}V^b=0$) y se interpreta como un tensor que mide cuánto falla $V^a$ en ser transportado paralelamente a lo largo de la congruencia, es decir, describe cuánto se desvían geodésicas cercanas en permanecer paralelas. En términos de este tensor se pueden dar definiciones análogas a las dichas previamente, quedando así: \textbf{tensor de vorticidad}: $\omega_{ab}=B_{[ab]}$, \textbf{escalar de expansión}: $\theta=B^{ab}h_{ab}$, \textbf{tensor de corte}: $\sigma_{ab}=B_{(ab)}-\frac{1}{3}\theta h_{ab}$. Cabe destacar que ambas definiciones son análogas, siendo esta última la presentada muchas veces en la bibliografía. Estas tres últimas definiciones (y no tanto el tensor de expansión $\theta_{ab}$) son de importancia ya que generan la descomposición de $B_{ab}$. Es decir, podemos descomponer a $B_{ab}$ como una parte antisimétrica, una parte simétrica, y una parte simétrica de traza nula:

\[B_{ab}=\frac{1}{3}\theta h_{ab}+\sigma_{ab}+\omega_{ab} \]

De manera sencilla se puede probar que los tensores definidos previamente son tensores espaciales: 

\[ \theta_{ab}V^a=\sigma_{ab}V^a=\omega_{ab}V^a=0 \]

A su vez, se puede ver que el escalar de expansión definido así, no es otra cosa más que la divergencia de $V^a$

\[\theta=\theta_{ab}g^{ab}=B^{ab}h_{ab}=h_{ab}\nabla^bV^a=\tensor{V}{^c_{;c}} \]

Para tratar de tener una idea más intuitiva sobre lo que representan los tensores previos, consideremos como ejemplo una pequeña esfera con partículas de prueba y veamos la evolución de dichas partículas respecto a las geodésicas centrales de las mismas. El escalar de expansión $\theta$ representa la parte con traza de $B_{ab}$ y describe el cambio en el volúmen de la esfera, lo cual le da el nombre al mismo. El tensor de corte $\sigma_{ab}$ representa la distorsión en la forma de la colección de las partículas de prueba, desde una esfera inicialmente hacia - por ejemplo - un elipsoide. El hecho de que $\sigma_{ab}$ sea la parte simétrica de $B_{ab}$ representa que una distorsión a lo largo de algún eje (supongamos $x$), es lo mismo que una distorsión a lo largo de $-x$. Finalmente, el tensor de vorticidad $\omega_{ab}$ que representa la parte antisimétrica de $B_{ab}$, describe la rotación alrededor de algún eje; por ejemplo, las componentes $xy$ del tensor describen la rotación alrededor del eje $z$. 

La evolución de la congruencia viene dada por la derivada covariante de los tensores de corte, expansión y vorticidad a lo largo de las geodésicas de la congruencia. Para ver esto más en detalle, calculemos la derivada covariante del tensor $B_{ab}$ a lo largo de las geodésicas de la congruencia:

$$
\frac{dB_{ab}}{d\tau}\equiv V^c\nabla_cB_{ab}=V^c\nabla_c\nabla_bV_a=V^c\nabla_b\nabla_cV_a+V^c\tensor{R}{^d_{abc}}V_d
$$
$$
=\nabla_b(V^c\nabla_cV_a)-(\nabla_bV^c)(\nabla_cV_a)-R_{dabc}V^cV^d
$$
\begin{equation}\label{Bab}
    =-\tensor{B}{^c_b}B_{ac}-R_{dabc}V^cV^d
\end{equation}



Tomándole traza a esta ecuación y usando las propiedades enunciadas previamente, se obtiene la ecuación

\begin{equation}\label{Raychaudhuri eq}
     \frac{d\theta}{d\tau}=-\frac{1}{3}\theta^2-\sigma_{ab}\sigma^{ab}+\omega_{ab}\omega^{ab}-R_{ab}V^aV^b
\end{equation}

que es la conocida como \textbf{Ecuación de Raychaudhuri}. La misma cumple un rol fundamental a la hora de probar los Teoremas de Singularidades y, notemos que, es una ecuación puramente geométrica, no se usan hipótesis sobre el espacio-tiempo ni sobre la distribución de materia en el mismo.


Si en vez de haberle tomado la traza a la ecuación (\ref{Bab}), nos hubiésemos quedado con la parte simétrica y antisimétrica de la misma, se obtiene otras dos ecuaciones a saber: en primer lugar, si nos quedamos con la parte simétrica de traza nula, se obtiene que

$$
\frac{d\sigma_{ab}}{d\tau}=-\frac{2}{3}\theta\sigma_{ab}-\sigma_{ac}\tensor{\sigma}{^c_b}-\omega_{ac}\tensor{\omega}{^c_b}+\frac{1}{3}h_{ab}(\sigma_{cd}\sigma^{cd}-\omega_{cd}\omega^{cd})+
$$
$$
+C_{cbad}V^cV^d+\frac{1}{2}\tilde{R}_{ab}
$$

donde $\tilde{R}_{ab}=h_{ac}h_{bd}R^{cd}-\frac{1}{3}h_{ab}h_{cd}R^{cd}$  es la parte con traza nula (y espacial) de $R_{ab}$ y $C_{cbad}$ es el tensor de Weyl. En segundo lugar, si nos quedamos con la parte antisimétrica, obtenemos que

$$
\frac{d\omega_{ab}}{d\tau}=-\frac{2}{3}\theta\omega_{ab}+\tensor{\sigma}{_a^c}\omega_{bc}-\tensor{\sigma}{_b^c}\omega_{ac}
$$


Estas tres ecuaciones representan la evolución de los tensores de corte, expansión y vorticidad a lo largo de geodésicas de la congruencia. A pesar de que las dos últimas ecuaciones no se las usa frecuentemente tanto como así sucede con la Ec. de Raychaudhuri (\ref{Raychaudhuri eq}), por completitud las hemos mostrado y además porque, en el caso de la evolución del tensor de vorticidad, se obtiene una conclusión a simple vista: si $\omega_{ab}=0$ inicialmente, entonces se mantendrá nulo a lo largo de la congruencia. Nos concentaremos de ahora en más solamente en al Ec. de Raychaudhuri. 

Notemos que como los tensores de corte y vorticidad son espaciales ($\omega_{ab}V^a=\sigma_{ab}V^a=0$), ambos tensores cumplen que

\begin{align*}
    \sigma_{ab}\sigma^{ab}\geq0&  &\omega_{ab}\omega^{ab}\geq0
\end{align*}

A su vez, el último término de la Ec. de Raychaudhuri se relaciona directamente con las Ec. de Einstein. De dichas ecuaciones sabemos que

\begin{equation*}
    R_{ab}V^aV^b=k\left(T_{ab}-\frac{1}{2}Tg_{ab}\right)V^aV^b
\end{equation*}

Si se satisface la condición de energía fuerte (SEC) entonces el último término de la Ec. de Raychaudhuri resulta negativo ya que $R_{ab}V^aV^b\geq0$ si es el caso en el que se cumplen las ecuaciones de Einstein y la SEC. Por lo tanto, si consideramos que la congruencia cumple $\omega_{ab}=0$ tal como sería el caso de una congruencia ortogonal a una hipersuperficie espacial, y asumiendo que se cumplen las Ec. de Einstein con la SEC, entonces la ecuación de Raychaudhuri implica que

\[\frac{d\theta}{d\tau}+\frac{1}{3}\theta^2\leq0 \]


Esta última ecuación se puede reescribir como

\[-\frac{1}{\theta^2}\frac{d\theta}{d\tau}=\frac{d}{d\tau}(\theta^{-1})\geq\frac{1}{3} \]

lo cual integrando se obtiene que

\begin{equation}\label{eq theta}
     \theta^{-1}(\tau)\geq \theta_0^{-1}+\frac{1}{3}\tau
\end{equation}

donde $\theta_0$ es el valor inicial de $\theta$. Si $\theta_0\leq 0$, entonces de (\ref{eq theta}) se deduce que $\theta^{-1}$ debe pasar por cero para cierto tiempo finito. Pero esto es lo mismo que decir que por lo tanto $\theta$ debe diverger en un tiempo propio $\tau\leq3/|\theta_0|$. Este último resultado se expresa de manera más formal en el siguiente Lema

\begin{lemma}\label{lema Raychaudhuri}
Sea $V^a$ el campo tangente de una congruencia de geodésicas temporales, ortogonal a una hipersuperficie espacial ($\omega_{ab}=0$). Si $R_{ab}V^aV^b\geq0$ - como sería el caso en el que se cumplan las Ec. de Einstein y la SEC - y si el escalar de expansión alcanza un valor negativo $\theta_0$ en algún punto de alguna geodésica de la congruencia, entonces $\theta$ tiende a $-\infty$ a lo largo de esa geodésica, en un tiempo propio $\tau\leq3/|\theta_0|$.
\end{lemma}

Hay que notar que la singularidad expresada por el Lema previo representa una singularidad en la congruencia de geodésicas y no necesariamente en la estructura del espacio-tiempo. Esto podría ser el caso de cáusticas que se cruzan para alguna congruencia en alguna superficie. Sin embargo, a pesar de que el Lema no establece una singularidad en la estructura del espacio-tiempo, con el agregado de ciertas consideraciones globales sí se puede establecer una singularidad en el espacio-tiempo. Estas consideraciones serán tomadas en cuenta más adelante, viendo ahora el caso de congruencias de geodésicas nulas.





%%%%%%%%%   Geodésicas nulas

    
\subsection{Geodésicas nulas}\label{geodesica nulas subsec}


En la sección \ref{geodesicas temporales} hemos tomado un campo de vectores $V^a$ tangentes a las geodésicas temporales, normalizados según $g_{ab}V^aV^b=-1$, y hemos estudiado la evolución del tensor espacial $B_{ab}=\nabla_bV_a$. Si se quiere hacer lo mismo para geodésicas nulas - i.e: estudiar la evolución de vectores en una superficie 3-dimensional normal al campo de vectores tangentes - nos encontramos con un problema y es que el vector tangente a curvas geodésicas es normal a si mismo ($g_{ab}k^ak^b=0$ para $k^a$ nulo). Para el caso de geodésicas nulas estaremos interesados, pues, en la evolución de vectores en una superficie 2-dimensional de vectores  normales al campo vectorial tangente de geodésicas nulas $k^a$ (las cuales vamos a suponer que están parametrizadas con parámetro afín $\lambda$). Para poder realizar un estudio de la congruencia de geodésicas nulas, vamos a tomar un vector nulo auxiliar $l^a$ tal que apunte en la dirección espacial opuesta a $k^a$, normalizado según 

\begin{align*}
    l^al_a=0&   &l^ak_a=-1&
\end{align*}

Además, el nuevo vector será transportado paralelamente 

\[ k^a\nabla_al^b=0 \]

ya que el transporte paralelo preserva el producto interno. La elección del nuevo vector $l^a$ es arbitraria, pero veremos que las cantidades relevantes son independientes de dicho vector. Llamaremos a la superficie 2-dimensional definida por los vectores normales como $T_\perp$, que consiste en los vectores $V^a$ ortogonales a $k^a$ y $l^a$:

\[ T_\perp=\{V^a \mid V^ak_a=0, V^al_a=0\} \]


Por lo tanto, el objetivo será el estudio de la evolución de vectores de desviación que están en dicha superficie, que representan una familia de geodésicas nulas cercanas. De manera análoga a la sección anterior, definiremos el operador de proyección $\hat{h}_{ab}$ como

\[ \hat{h}_{ab}=g_{ab}+k_al_b+k_bl_a \]


el cual actuará como la métrica cuando se aplica a vectores $V^a \in T_\perp$ y será nulo actuando sobre vectores proporcionales a $k^a$ o $l^a$. Algunas propiedades que satisface $\hat{h}_{ab}$ son las siguientes


\begin{enumerate}[1)]
    \item $\tensor{\hat{h}}{^a_b}V^b=V^a$
    \item $\tensor{\hat{h}}{^a_b}\tensor{\hat{h}}{^b_c}=\tensor{\hat{h}}{^a_c}$
    \item $k^c\nabla_c\tensor{\hat{h}}{^a_b}=0$
    \item $\tensor{\hat{h}}{^a_a}=\tensor{g}{^a_a}+2k^al_a=2$
\end{enumerate}

Al igual que antes, llamaremos $\tensor{B}{^a_b}=\nabla_bk^a$ al tensor que mide cuánto falla $V^a$ en ser transportado paralelamente:

\[ \frac{dV^a}{d\lambda}=k^b\nabla_bV^a=\tensor{B}{^a_b}V^b  \]

Sin embargo, veremos que para el caso de geodésicas nulas alcanza con estudiar el comportamiento de la parte contenida en la proyección, $\tensor{\hat{B}}{^a_b}=\tensor{\hat{h}}{^a_c}\tensor{B}{^c_d}\tensor{\hat{h}}{^d_b}$, y no todo el tensor $B_{ab}$:


$$
\frac{dV^a}{d\lambda}=k^b\nabla_bV^a=k^b\nabla_b(\tensor{\hat{h}}{^a_c}V^c)=\tensor{\hat{h}}{^a_c}k^b\nabla_bV^c
$$
$$
=\tensor{\hat{h}}{^a_c}\tensor{B}{^c_b}V^b=\tensor{\hat{h}}{^a_c}\tensor{B}{^c_b}\tensor{\hat{h}}{^b_d}V^d
$$
$$
=\tensor{\hat{B}}{^a_d}V^d
$$

lo cual demuestra lo dicho. Al igual que antes, descompondremos al tensor como

\[ \hat{B}_{ab}=\frac{1}{2}\theta\hat{h}_{ab}+\hat{\sigma}_{ab}+\hat{\omega}_{ab} \]

en donde el factor $1/2$ en lugar de $1/3$ tal como era en el caso de geodésicas temporales viene por el hecho de que $T_\perp$ es una superficie 2-dimensional, donde $\tensor{\hat{h}}{^a_a}=2$. Los tensores de vorticidad, expansión y corte se definen de forma análoga a la sección anterior como

$$
\tensor{\hat{\omega}}{_{ab}}=\hat{B}_{[ab]}
$$
$$
\theta=\hat{h}^{ab}\hat{B}_{ab}=\tensor{\hat{B}}{^a_a}
$$
$$
\tensor{\hat{\sigma}}{_{ab}}=\hat{B}_{(ab)}-\frac{1}{2}\theta\hat{h}_{ab}
$$

Aquí hemos denotado $\theta$ y no $\hat{\theta}$ por el hecho de que el mismo no depende de $l^a$: 

$$
\theta=\hat{h}^{ab}\hat{B}_{ab}=\hat{h}^{ab}B_{ab}=g^{ab}B_{ab}
$$

en donde en la segunda igualdad usamos que $\hat{h}^{ab}\tensor{\hat{h}}{^c_b}=\hat{h}^{ac}$ y en la tercera igualdad que $k^aB_{ab}=k^bB_{ab}=0$. Habiendo definido las cantidades previas, veamos ahora la evolución de $\hat{B}_{ab}$:

$$
\frac{d\hat{B}_{ab}}{d\lambda}=k^c\nabla_c\hat{B}_{ab}=k^c\nabla_c(\tensor{\hat{h}}{^d_a}\nabla_dk_f\tensor{\hat{h}}{^f_b})
$$
$$
=\tensor{\hat{h}}{^d_a}\tensor{\hat{h}}{^f_b}k^c\nabla_c\nabla_dk_f
$$
$$
=-\tensor{\hat{h}}{^d_a}\tensor{\hat{h}}{^f_b}(\tensor{B}{_d^c}B_{fc}+R_{dgfc}k^gk^c)
$$
\begin{equation}\label{Bab null}
=-\tensor{\hat{B}}{_a^c}\hat{B}_{bc}-\tensor{\hat{h}}{^d_a}\tensor{\hat{h}}{^f_b}R_{dgfc}k^gk^c
\end{equation}


Tomándole traza a la ecuación (\ref{Bab null}) se obtiene 

\begin{equation}\label{Raychaudhuri null}
\frac{d\theta}{d\lambda}=-\frac{1}{2}\theta^2-\hat{\sigma}_{ab}\hat{\sigma}^{ab}+\hat{\omega}_{ab}\hat{\omega}^{ab}-R_{ab}k^ak^b
\end{equation}

que es análoga a la ecuación (\ref{Raychaudhuri eq}) pero para geodésicas nulas. Tal como se dijo previamente, esta ecuación no depende de $l^a$: $\theta$ ya hemos demostrado que no depende de $l^a$ y, a pesar de que $\hat{\sigma}_{ab}$ y $\hat{\omega}_{ab}$ dependen de $l^a$, $\hat{\sigma}_{ab}\hat{\sigma}^{ab}$ y $\hat{\omega}_{ab}\hat{\omega}^{ab} $ no lo hacen, como se puede verificar fácilmente usando la definición de $\hat{\omega}_{ab}$ y $\hat{\sigma}_{ab}$, y que $k^a$ es un vector nulo. Si en vez de tomarle traza a (\ref{Bab null}), nos hubiésemos quedado con la parte simétrica y con la parte antisimétrica, se obtienen ecuaciones para la evolución de $\hat{\omega}_{ab}$ y $\hat{\sigma}_{ab}$:

$$
\frac{d\hat{\sigma}_{ab}}{d\lambda}=-\theta\hat{\sigma}_{ab}-\tensor{\hat{h}}{^d_a}\tensor{\hat{h}}{^c_b}C_{dfcg}k^fk^g
$$
$$
\frac{d\hat{\omega}_{ab}}{d\lambda}=-\theta\hat{\omega}_{ab}
$$

Estas dos últimas ecuaciones han sido mostradas simplemente por completitud. La verdadera ecuación importante aquí es (\ref{Raychaudhuri null}) y no tanto la evolución del tensor de corte y de vorticidad, al igual de lo que sucedía con las ecuaciones análogas en geodésicas temporales. De las Ec. de Einstein, como $k^a$ es un vector nulo, se deduce que


$$
R_{ab}k^ak^b=k\left( T_{ab}-\frac{1}{2}Tg_{ab} \right)k^ak^b
$$
$$
=kT_{ab}k^ak^b
$$

y por lo tanto el último término de (\ref{Raychaudhuri null}) resulta negativo si se cumple la NEC. Notemos que la NEC es una condición menos restrictiva que la SEC, por lo tanto geodésicas nulas tienden a converger a cáusticas bajo condiciones más generales que las geodésicas temporales. A su vez, argumentos análogos a la sección anterior permiten enunciar un lema equivalente al \ref{lema Raychaudhuri} pero para geodésicas nulas:

\begin{lemma}
 Sea $k^a$ el campo tangente de una congruencia de geodésicas nulas con $\hat{\omega}_{ab}=0$ (hipersuperficie ortogonal). Si $R_{ab}k^ak^b\geq0$ - como sería el caso en el que se cumplan las Ec. de Einstein y la NEC o la SEC - y si el escalar de expansión alcanza un valor negativo $\theta_0$ en algún punto de alguna geodésicas de la congruencia, entonces $\theta$ tiende a $-\infty$ a lo largo de esa geodésicas, con longitud afín $\lambda\leq 2/|\theta_0|$.
\end{lemma}







%%%%%%%%%%%%%%%%%%%%%%%%   Puntos conjugados

    
\section{Puntos conjugados}\label{pts conjugados}



En la sección \ref{ec. desviacion geodesica} hemos dado una definición de \textit{puntos conjugados}. Una idea intuitiva que podemos pensar es en que dos puntos son conjugados si una geodésica infinetesimalmente cerca interseca $\gamma$ en ambos puntos en cuestión, donde $\gamma$ es la geodésica que contiene a ambos puntos. En la figura \ref{fig pts conjugados} se ilustra lo dicho.


\begin{figure}[h]
    \centering
    \includegraphics[width=0.2\textwidth]{./img/Cap4/ptsconjugados.png}
    \caption{Ilustración de puntos conjugados $p$ y $q$ a lo largo de la geodésica $\gamma$.}
    \label{fig pts conjugados}
\end{figure}



El estudio de puntos conjugados es de especial interés ya que caracterizan cuándo una geodésica falla en ser la curva que extrema la longitud de dos puntos en un espacio-tiempo. Al igual que en la sección anterior, haremos la distinción entre geodésicas temporales y geodésicas nulas, comenzando por las primeras (por simplicidad) y dando los resultados análogos para las segundas.

Sea $\gamma$ una geodésica temporal con vector tangente $V^a$, y sea $p\in \gamma$. Consideremos la congruencia de todas las geodésicas temporales que pasan por $p$ y sean $e^a_1, e^a_2, e^a_3$ una base ortonormal de vectores espaciales (ortogonales a $V^a$), propagados paralelamente a lo largo de $\gamma$ ($V^c\nabla_ce^a_k=0$, $k=1,2,3$). Sea $\eta^a$ un vector de desviación en la congruencia; $\eta^a$ satisface la Ec. de Jacobi, por lo que a lo largo de $\gamma$ vale 

$$
\frac{d^2\eta^a}{d\tau^2}=-\tensor{R}{_{cbd}^a}\eta^bT^cT^d
$$

El valor de $\eta^a$ a tiempo $\tau$ depende linealmente de las condiciones iniciales $\eta^a(0)$ y $d\eta^a(0)/d\tau$. Como por construcción de la congruencia $\eta^a(0)=0$, entonces debe ser 


\begin{equation}\label{Aab}
\eta^a(\tau)=\tensor{A}{^a_b}(\tau)\frac{d\eta^b}{d\tau}(0)
\end{equation}


Reemplazando en la ecuación de Jacobi se obtiene que $\tensor{A}{^a_b}(\tau)$ debe satisfacer

$$
\frac{d^2\tensor{A}{^a_b}}{d\tau^2}=-\tensor{R}{_{cfd}^a}\tensor{A}{^f_b}T^cT^d
$$

Dadas las condiciones iniciales para $\eta^a$, es fácil ver que las condiciones iniciales para $\tensor{A}{^a_b}$ serán pues $\tensor{A}{^a_b}(0)=0$ y $d\tensor{A}{^a_b}(0)/d\tau=\tensor{\delta}{^a_b}$.

Un punto $q\in \gamma$ es conjugado a $p$ si y solo si el vector de desviación se anula en $q$ y no es idénticamente nulo. De la ecuación (\ref{Aab}) se deduce que esto ocurrirá si y solo si det$(\tensor{A}{^a_b})=0$ en $q$ (entre $p$ y $q$, el det$(\tensor{A}{^a_b})\neq0$ por lo que existe inversa de $\tensor{A}{^a_b}$).

La matriz $\tensor{A}{^a_b}$ puede ser relacionada con el tensor $B_{ab}=\nabla_bV_a$ definido previamente de la siguiente manera: por un lado 


$$
\frac{d\eta^a}{d\tau}=V^b\nabla_b\eta^a=\eta^b\nabla_bV^a=\tensor{B}{^a_b}\eta^b=\tensor{B}{^a_b}\tensor{A}{^b_c}\frac{d\eta^c}{d\tau}(0)
$$

en donde se ha usado que $V$ y $\eta$ conmutan (\ref{conmutador}). Por otro lado, usando la ecuación (\ref{Aab}), se sigue que

$$
\frac{d\eta^a}{d\tau}=\frac{d\tensor{A}{^a_b}}{d\tau}\frac{d\eta^b}{d\tau}(0)
$$

Igualando las ecuaciones se obtiene que

$$
\frac{d\tensor{A}{^a_b}}{d\tau}=\tensor{B}{^a_c}\tensor{A}{^c_b}
$$

que en forma matricial es lo mismo que 

$$
B=\frac{dA}{d\tau}A^{-1}
$$

Tomándole traza a esta última ecuación, se obtiene que

$$
\text{tr} B=\theta=\text{tr}\left[\frac{dA}{d\tau}A^{-1}\right]=\frac{d}{d\tau}(\text{log}|\text{det}A|)
$$

De aquí se deduce que det$(A)\rightarrow0$ en $q$ si y solo si $\theta\rightarrow -\infty$ en $q$ y por lo tanto la condición necesaria y suficiente para que $q$ sea un punto conjugado a $p$ es que para la congruencia de geodésicas temporales que emanan de $p$ sea $\theta\rightarrow -\infty$ en $q$. Es posible probar \citep{1984ucp..book.....W} que la congruencia de geodésicas cumple $\omega_{ab}=0$, y mediante el uso de la Ec. de Raychaudhuri se puede concluir la siguiente proposición:

\begin{proposition}\label{prop pts conjugados}
Sea $(M.g_{ab})$ un espacio-tiempo que satisface $R_{ab}V^aV^b\geq0$ para todo vector temporal $V^a$ - tal como sería el caso en que se cumplan las Ec. de Einstein y la SEC. Sea $\gamma$ una geodésica temporal y sea $p\in \gamma$. Si el escalar de expansión $\theta$ para la congruencia de geodésicas temporales que emanan al futuro de $p$, alcanza un valor negativo $\theta_0$ en un punto $r\in \gamma$, entonces en un tiempo propio $\tau\leq 3/|\theta_0|$ desde $r$ a lo largo de $\gamma$, existe un punto $q$ conjugado a $p$, suponiendo que $\gamma$ sea lo suficientemente extensa.
\end{proposition}

La existencia de puntos conjugados puede ser probada mediante hipótesis más débiles que las de la proposición anterior. Para ello, definiremos lo siguiente: un espacio-tiempo $(M,g_{ab})$ se dice que cumple la \textbf{condición genérica temporal} si cada geodésica temporal posee al menos un punto donde $R_{abcd}V^aV^d\neq0$. De esta manera se puede enunciar la siguiente proposición:

\begin{proposition}\label{pts conjugados cgt}
Sea $(M,g_{ab})$ un espacio-tiempo que cumple la condición genérica temporal y supongamos que $R_{ab}V^aV^b\geq0$ para todo vector temporal $V^a$, entonces cada geodésica temporal completa posee un par de puntos conjugados.
\end{proposition}

Cabe destacar que aunque la condición genérica temporal no se satisfaga necesariamente para espacios-tiempos especiales, es una condición razonable para modelos físicamente razonables de espacios-tiempos ``genéricos" (polvo, campo e.m, etc). A su vez, se puede generalizar la Proposición \ref{prop pts conjugados} para hipersuperficies suaves y espaciales. Para ello definiremos  la \textit{curvatura extrínseca}: sea $\Sigma$ una hipersuperficie espacial y suave, y sea $V^a$ el campo tangente unitario a la congruencia de geodésicas temporales ortogonales a $\Sigma$. Se define la \textbf{curvatura extrínseca} $K_{ab}$ como 

$$
K_{ab}=\nabla_aV_b=B_{ba}
$$

Ciertas propiedades que cumple la curvatura extrínseca son

\begin{enumerate}[1)]  
    \item $K_{ab}V^a=K_{ab}V^b=0$, i.e: es puramente espacial
    \item $K_{ab}=K_{ba}$
    \item $K=\text{tr}K_{ab}=\text{tr}B_{ab}=\theta$
\end{enumerate}

La curvatura extrínseca $K_{ab}$ mide cuánto se ``tuerce" la hipersuperficie $\Sigma$ en el espacio-tiempo $(M,g_{ab})$, pensando a $\Sigma$ como una subvariedad 3-dimensional embebida en $(M,g_{ab})$. Otra forma de ver esto es si se piensa en la métrica espacial $h_{ab}$, entonces $K_{ab}$ mide el cambio de la métrica espacial cuando uno se mueve a lo largo de la congruencia. De manera análoga a dos puntos conjugados, se puede definir puntos conjugados a una hipersuperficie y llegar a resultados análogos a la Proposición \ref{prop pts conjugados}:

\begin{proposition}\label{prop pto conjugado}
Sea $(M,g_{ab})$ un espacio-tiempo que cumple $R_{ab}V^aV^b\geq0$ para todo vector temporal $V^a$ (tal como sería el caso en el que valgan las Ec. de Einstein y se satisfaga la SEC). Sea $\Sigma$ una hipersuperficie espacial con $K=\theta\leq0$ en algún punto $p\in \Sigma$. Entonces en un tiempo propio $\tau\leq3/|K|$ existe un punto $q$ conjugado a $\Sigma$ a lo largo de una geodésica $\gamma$ ortogonal a $\Sigma$ y que pasa por $p$, suponiendo que $\gamma$ sea lo suficientemente extensa.
\end{proposition}

La existencia de puntos conjugados se relaciona íntimamente con la longitud de arco de geodésicas (temporales). Sean $p,q\in M$ y consideremos una familia uniparamétrica de curvas temporales suaves $\lambda_s(t):[a,b]\rightarrow M$ de $p$ a $q$ tal que para todo $s$ se tiene $\lambda_s(a)=p$ y $\lambda_s(b)=q$. Sea $T^a$ el vector tangente y $X^a$ el vector de desviación, entonces la longitud de la curva (o tiempo propio) viene dada por

$$
\tau(s)=\int\limits_a^b(-T^aT_a)^{1/2}dt
$$

Es posible probar \citep{1984ucp..book.....W} que la variación de arco del tiempo propio otorga la siguiente ecuación: 


$$
\frac{d^2\tau}{ds^2}\bigg\rvert_{s=0}=\int\limits_a^b X^b\{T^c\nabla_c(T^a\nabla_aX_b)+\tensor{R}{_{cab}^d}X^cT^aT_d\}dt
$$

donde notemos que el término integral es precisamente la ecuación de Jacobi. En base a esto se puede probar el siguiente teorema (\citep{1984ucp..book.....W} Teorema 9.3.3):

\begin{theorem}
Sea $\gamma$ una curva temporal suave que une los puntos $p,q\in M$. La condición necesaria y suficiente para que $\gamma$ maximice el tiempo propio entre $p$ y $q$ sobre todas las variaciones uniparamétricas, es que $\gamma$ sea una geodésica sin puntos conjugados entre $p$ y $q$.
\end{theorem}

En el caso en que existan puntos conjugados a una geodésica temporal $\gamma$, lo que ocurre es que es posible dar una variación de $\gamma$ tal que $\gamma$ no es un máximo local. Es decir, cualquier otra curva de la variación que pasa por $p$, y pasa infinetesimal cerca de $q$, tiene el mismo (o mayor) tiempo propio. El mismo teorema previo se puede enunciar para hipersuperficies espaciales de forma análoga:

\begin{theorem}\label{pts conjugados teo temporal}
Sea $\gamma$ una curva temporal suave que une un punto $q\in M$ con un punto $p$ en una hipersuperficie espacial suave $\Sigma$. La condición necesaria y suficiente para que $\gamma$ maximice el tiempo propio entre $q$ y $\Sigma$ sobre todas las variaciones uniparamétricas es que $\gamma$ sea una geodésica ortogonal a $\Sigma$ sin puntos conjugados entre $\Sigma$ y $q$.
\end{theorem}


Cabe destacar que las geodésicas que extreman el tiempo propio son geodésicas sin quiebres; es decir, con tangente continuo. De lo contrario, sería posible tomar una curva estilo ``zigzag" (o inclusive una estilo  ``fractal") la cual siempre posea tiempo propio mayor o igual que cualquier otra geodésica. En la siguiente sección se hará una discusión más en detalle sobre este tipo de curvas.

Una vez visto el caso de geodésicas temporales, veamos ahora el caso con geodésicas nulas generalizando así para curvas causales. De la ecuación de desviación geodésica se deduce que para cualquier campo de Jacobi $\eta^a$ en una geodésica nula $\mu$ con tangente $k^a$ cumple 


$$
k^c\nabla_c[k^b\nabla_b(k^a\eta_a)]=0
$$


lo cual implica que $\eta^a$ se anula en $p$ y $q$ si y solo si $k^a\eta_a=0$ sobre toda la geodésica. Más aún, si $\eta^a$ es un campo de Jacobi, entonces también lo es $\tilde{\eta}^a=\eta^a+(a+b\lambda)k^a$, con $a$ y $b$ constantes, y por lo tanto $p$ y $q$ serán conjugados si y solo si existe un campo de Jacobi $\tilde{\eta}^a$ no idénticamente nulo pero que se anula en $p$ y $q$. Esto equivale a decir que $\eta^a$ difiere de cero por un  múltiplo de $k^a$ en $p$ y $q$. De esta forma, los puntos $p,q\in \mu$ serán conjugados a lo largo de la geodésica nula si y solo si un vector $\hat{\eta}^a$ en $T_\perp$ cumple la ecuación de desviación geodésica y se anula en $p$ y $q$. Todos los $\hat{\eta}^a$ que se anulan en $p$, surgen como vectores de desviación de cualquier congruencia de geodésicas nulas que contiene la familia 2-dimensional de geodésicas nulas que emergen de $p$. De esta forma, usando argumentos análogos al caso de geodésicas temporales, $q$ será conjugado a $p$ si y solo si el escalar de expansión $\theta$ (de la congruencia de geodésicas nulas) tiende a $-\infty$ en $q$. Así se puede obtener un resultado similar al dado en la Proposición \ref{prop pts conjugados} pero para geodésicas nulas

\begin{proposition}\label{pts conjugados null}
Sea $(M,g_{ab})$ un espacio-tiempo que satisface $R_{ab}k^ak^b\geq0$ para todo vector nulo $k^a$ - como sería el caso en el que valgan las Ec. de Einstein y se cumpla la NEC. Sea $\mu$ una geodésica nula y sea $p\in \mu$. Si el escalar de expanión $\theta$ para la congruencia de geodésicas nulas que emanan de $p$ alcanza un valor negativo $\theta_0$ en un punto $r\in \mu$, entonces con longitud afín $\lambda\leq2/|\theta_0|$ de $r$, existe un punto $q$ conjugado a $p$ a lo largo de $\mu$, suponiendo que $\mu$ sea lo suficientemente extensa.
\end{proposition}


Un resultado análogo a la Proposición \ref{pts conjugados cgt} se puede expresar definiendo lo siguiente: un espacio-tiempo $(M,g_{ab})$ se dice que satisface la \textbf{condición genérica nula} si cada geodésica nula posee al menos un punto donde $R_{ab}k^ak^b\neq0$.

\begin{proposition}
Sea $(M,g_{ab})$ un espacio-tiempo que cumple la condición genérica nula, y supongamos que $R_{ab}k^ak^b\geq0$ para todo vector nulor $k^a$, entonces cada geodésica nula completa posee un par de puntos conjugados.
\end{proposition}


Vimos que para geodésicas temporales el hecho de que una curva posea puntos conjugados se refleja en que una geodésica deja de ser la curva maximal: una curva que pasa infinetesimalmente cerca puede tener tiempo propio igual o mayor. En geodésicas nulas, el hecho de que las mismas posean puntos conjugados, se refleja en que ahora la geodésica que une dos puntos deja de ser nula y se convierte en una curva temporal, para alguna curva que pasa infinetesimalmente cerca. Un resultado importante (cuya demostración se puede ver en \citep{1975lsss.book.....H} Proposición 4.5.11) que refleja lo dicho es el siguiente:

\begin{theorem}
Sea $\mu$ una curva causal suave y sean $p,q\in \mu$. Entonces $\mu$ es una geodésica nula sin puntos conjugados entre $p$ y $q$ si y solo si no existe una familia uniparamétrica suave de curvas causales $\lambda_s$ que unen $p$ y $q$ con $\lambda_0=\mu$ y $\lambda_s$ temporal para todo $s>0$ (es decir, no se puede deformar de manera suave a $\mu$ de forma tal que sea una curva temporal). 
\end{theorem}



A su vez, la Proposición \ref{pts conjugados null} se puede extender para superficies:

\begin{proposition}\label{Prop 9.3.9 Wald}
Sea $(M,g_{ab})$ un espacio-tiempo que cumple $R_{ab}k^ak^b\geq0$ para todo vector nulo $k^a$ - como sería el caso en el que valen las Ec. de Einstein y la SEC. Sea $S$ una superficie espacial 2-dimensional suave tal que el escalar de expansión $\theta$ alcanza un valor negativo $\theta_0$ en algún punto $p\in S$, y sea $\mu$ una geodésica nula ortogonal a $S$ que pasa por $p$. Entonces con longitud afín $\lambda\leq2/|\theta_0|$ existe un punto $q$ conjugado a $S$ a lo largo de $\mu$.
\end{proposition}


Un resultado análogo al Teorema \ref{pts conjugados teo temporal} pero que vale para geodésicas nulas se enuncia a continuación:

\begin{theorem}
Sea $S$ una superficie espacial 2-dimensional suave y sea $\mu$ una curva causal suave que une $S$ con un punto $q\in M$. Entonces la condición necesaria y suficiente para que $\mu$ no pueda ser deformada suavemente hacia una curva temporal (que una $S$ y $q$) es que $\mu$ sea una geodésica nula ortogonal a $S$ sin puntos conjugados entre $S$ y $q$.
\end{theorem}

Por último, daremos un resultado (cuya demostración se puede ver en \citep{1984ucp..book.....W} Teorema 9.3.11) que será usado a la hora de demostrar los Teoremas de singularidades:

\begin{theorem}\label{Teo 9.3.11 Wald}
Sea $(M,g_{ab})$ un espacio-tiempo globalmente hiperbólico y sea $K$ una superficie espacial 2-dimensional compacta y orientable, entonces cada punto en el borde de $I^+(K)$ (denotado $\dot{I}^+(K)$) está sobre una geodésica nula futuro directo ortogonal a $K$ y que no posee puntos conjugados entre $K$ y $p$.
\end{theorem}


%%%%%%%%%%%%%%%%%%%%%%%%   Existencia de curvas de longitud máxima

    
\section{Existencia de curvas de longitud máxima}

    


En la sección \ref{seccion Cpq} hemos definido el espacio de curvas causales que conectan dos puntos, $C(p,q)$. El objetivo en esta sección será mostrar la existencia de curvas de máxima longitud en espacios-tiempos globalmente hiperbólicos, usando la compacticidad de $C(p,q)$. En base a esto se podrá arribar a un resultado sobre la completitud de geodésicas, el cual es el último elemento necesario para, finalmente, enunciar y demostrar los Teoremas de Singularidades en el siguiente capítulo. 

Comencemos definiendo $\widetilde{C}(p,q)$: el subconjunto de las curvas suaves de $C(p,q)$, con la topología inducida por $C(p,q)$. Es posible probar que, defindo así, $\widetilde{C}(p,q)$ resulta denso en $C(p,q)$; es decir, cada curva causal continua puede ser expresada como el límite de una sucesión de curvas suaves temporales. Llamaremos $\tau$ a la longitud de la curva causal suave $\lambda$, entre los puntos $p,q\in M$, con tangente $T^a=(\partial/\partial t)^a$: 

$$
\tau[\lambda]=\int (-T^aT_a)^{1/2}dt
$$


Es posible ver que $\tau$ no es continua en $\widetilde{C}(p,q)$: siempre podemos tomar una curva temporal estilo zigzag (suavizada) que tenga longitud arbitrariamente cercana a cero, y que esté arbitrariamente cerca a cualquier curva temporal de $\widetilde{C}(p,q)$. Sin embargo, $\tau[\lambda]$ sí resulta \textit{semicontinua superior} en $\widetilde{C}(p,q)$. Esto es: dado $\epsilon>0$, para cada $\lambda\in \widetilde{C}(p,q)$ existe un entorno abierto $O\subset \widetilde{C}(p,q)$ de $\lambda$ tal que para todo $\lambda' \in O$, se tiene que $\tau[\lambda']\leq \tau[\lambda]+\epsilon$. Este resultado se formaliza en la siguiente proposición:

\begin{proposition}
Sea $(M,g_{ab})$ un espacio-tiempo fuertemente causal y $p,q\in M$ con $q\in I^+(p)$, entonces $\tau$ es semicontinua superior en $\widetilde{C}(p,q)$.
\end{proposition}
\begin{proof}
Ver \citep{1984ucp..book.....W} Proposición 9.4.1.
\end{proof}




De esta forma, dado que $\tau$ es semicontinua superior en $\widetilde{C}(p,q)$, se puede extender hacia una función semicontinua superior en $C(p,q)$ de la siguiente manera: sea $\mu\in C(p,q)$ y $O\subset C(p,q)$ un entorno abierto de $\mu$, se define

$$
T[O]=\text{sup}\{\tau[\lambda] \mid \lambda\in O\cap \widetilde{C}(p,q)\}
$$
$$
\tau[\mu]=\text{inf}\{T[O] \mid  \text{O es un entorno abierto de $\mu$}\}
$$


donde $T[O]$ indica la longitud más grande $\tau$ en el entorno abierto $O$ y, luego, $\tau[\mu]$ es el valor de $\tau$ a medida que achicamos el entorno $O$.

En la sección anterior, \ref{pts conjugados}, vimos que la condición necesaria y suficiente para que una curva suave maximizara la longitud entre dos puntos (o un punto y una hipersuperficie) era que la misma sea una geodésica sin puntos conjugados. Sin embargo, al extender la definición de $\tau$ a curvas continuas, podría pasar que una curva continua pero no suave entre dos puntos (o un punto y una hipersuperficie) posea longitud mayor o igual que la geodésica en cuestión. Esta posibilidad no será tenida en cuenta y los siguientes resultados lo justifican.




\begin{theorem}
Sea $(M,g_{ab})$ un espacio-tiempo globalmente hiperbólico y sean $p,q\in M$ con $q\in J^+(p)$, entonces existe una curva $\gamma\in C(p,q)$ en donde $\tau$ alcanza su máximo valor sobre ella.
\end{theorem}
\begin{proof}
$C(p,q)$ es compacto y $\tau$ es semicontinua superior. Usando que las funciones continuas en un compacto están acotadas y alcanzan su máximo valor (y que vale para funciones semicontinuas también) entonces $\tau$ resulta acotada y posee su valor máximo en $C(p,q)$.
\end{proof}



Más aún, se puede demostrar que esa curva que hace máximo a $\tau$ es, de hecho, la geodésica que maximiza $\tau$ sobre $C(p,q)$:


\begin{theorem}
Sea $(M,g_{ab})$ un espacio-tiempo globalmente hiperbólico y sean $p,q\in M$ con $q\in J^+(p)$. Una condición necesaria para que $\tau$ alcance su máximo valor en $\gamma\in C(p,q)$ es que $\gamma$ sea una geodésica sin puntos conjugados entre $p$ y $q$.
\end{theorem}
\begin{proof}
En un entorno convexo normal $U$, la única geodésica $\gamma$ que une causalmente dos puntos $r,s\in U$ tiene longitud estrictamente mayor que cualquier otra curva suave causal que una dichos puntos (Proposición 4.5.3 \citep{1984ucp..book.....W}). Sea $\gamma$ la geodésica que une $r,s\in U$ y sea $\mu$ una curva causal continua que une dichos puntos. Por (semi)continuidad superior, sabemos que 
$$
\tau[\mu]\leq\tau[\gamma]
$$
Sin embargo, consideremos que vale la igualdad $\tau[\mu]=\tau[\gamma]$, con $\mu\neq\gamma$. Sea $q$ un punto tal que $q\in \mu$ pero $q\notin \gamma$. Llamemos $\gamma_1,\gamma_2$ a las geodésicas que unen $r$ con $q$ y $q$ con $s$ respectivamente. Como $\gamma_1$ maximiza la longitud entre $r$ y $q$, y $\gamma_2$ maximiza la distancia entre $q$ y $s$, entonces debería ser
$$
\tau[\gamma_1]+\tau[\gamma_2]\geq\tau[\mu]=\tau[\gamma]
$$
Sin embargo, esto contradice el hecho de que $\gamma$ tiene estrictamente longitud mayor que cualquier otra curva suave entre $r$ y $s$ y, por lo tanto, se deduce que para cualquier entorno convexo normal, la única geodésica que une causalmente dos puntos tiene longitud estrictamente mayor que cualquier otra curva continua causal que una dichos puntos. En general, cualquier curva continua causal arbitraria que una dos puntos no puede ser una curva de máxima longitud entre dichos puntos al menos que sea geodésica, ya que si falla en ser geodésica en algún punto, se podría deformar a la curva en un entorno convexo normal de ese punto para así obtener una curva de longitud mayor.
\end{proof}


%\textcolor{blue}{(6.7.1 Hawking-Ellis)}
%\begin{proposition}
%Sea $(M,g_{ab})$ un espacio-tiempo globalmente hiperbólico y sean $p,q\in %M$ con $q\in J^+(p)$, entonces existe una geodésica causal que une $p$ y %$q$ y cuya longitud es mayor o igual que cualquier otra curva causal que %une dichos puntos.
%\end{proposition}

A su vez, se pueden generalizar estos últimos dos teoremas para hipersuperficies:

\begin{theorem}\label{teo existencia curva maxima long}
Sea $(M,g_{ab})$ un espacio-tiempo globalmente hiperbólico. Sea $p\in M$ y sea $\Sigma$ una superficie de Cauchy, entonces existe una curva $\gamma\in C(\Sigma,p)$ en donde $\tau$ alcanza su máximo valor.
\end{theorem}


\begin{theorem}
Sea $(M,g_{ab})$ un espacio-tiempo globalmente hiperbólico. Sea $p\in M$ y sea $\Sigma$ una hipersuperficie espacial suave y acronal. Una condición necesaria para que $\tau$ alcance su máximo valor sobre $\gamma\in C(\Sigma,q)$ es que $\gamma$ sea una geodésica ortogonal a $\Sigma$ sin puntos conjugados entre $\Sigma$ y $p$.
\end{theorem}




Con estos resultados se finaliza el capítulo y, ahora sí, ya estamos en condiciones de enunciar y demostrar los Teoremas de Singularidades en el siguiente capítulo.

