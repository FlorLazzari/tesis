\chapter{Geometría Diferencial, Topología y Relatividad General}\label{geo dif}



En el siguiente capítulo se presentan y se repasan brevemente los conceptos utilizados a lo largo de la tesis sobre geometría diferencial, topología y Relatividad General, a modo de resúmen sobre posibles tópicos dados en un curso básico de Relatividad General.



%%%%%%%%%%%%%%%%%%%   Geometría Diferencial y Topología


\section{Geometría Diferencial y Topología}

A continuación se presentan los conceptos matemáticos utilizados en la tesis sobre geometría diferencial y topología, siguiendo de referencia a \citep{1984ucp..book.....W,1980gmmp.book.....S}. A su vez, se excluyen ciertas definiciones y teoremas para dejar más amena la lectura. Para dichas exclusiones referirse al Apéndice \ref{esptopologicos}




%%%%%%%%%%%%%%%%%%%   Variedades diferenciales

    
\subsection{Variedades diferenciales}%\label{magnitud}

Dados dos conjuntos $M$ y $N$, se define el \textit{mapa} $\phi$ de $M$ a $N$ como la regla que asocia un elemento $x\in M$, un único elemento $y\in N$. Cuando el mapa es biyectivo y bicontinuo, se dice que es un \textit{homeomorfismo} (relación 1-1). Un espacio topológico de Hausdorff se dice una \textbf{variedad $M$} de dimensión $n$ si cada punto de $M$ tiene un entorno abierto el cual admite un homeomorfismo con un abierto de $\mathbb{R}^n$. La idea intuitiva es que, localmente, una variedad es un objeto geométrico que se asemeja a $\mathbb{R}^n$.

Una \textit{carta} en $M$ se define como el par $(U,\phi)$ donde $U\subseteq M$ y $\phi:M\rightarrow \mathbb{R}^n$ es un mapa biyectivo. Un \textit{atlas} se define como el conjunto de cartas que cubren toda la variedad. Por definición, un mapa asocia un punto $P\in M$ una n-upla $(x^1(P),...,x^n(P))$ en $\mathbb{R}^n$. A $x^1(P),...,x^n(P)$ se los denomina \textit{coordenadas} de $P$. Sean las cartas $(U,\phi_i$), $(V,\phi_j)$ donde $U, V\in M$ con $U\cap V\neq \emptyset$. Consideremos la función $\phi_j\circ\phi_i^{-1}$ que me relaciona los puntos $\phi_i(U\cap V)\subset\phi_i(U)$ con $\phi_j(U\cap V)\subset\phi_j(V)$. Si estas funciones y sus inversas son de clase $C^k$ diremos que las cartas son $C^k$-relacionadas. Si es posible construir un atlas donde cada carta sea $C^k$-relacionada con las demás diremos que es una \textit{variedad $C^k$} o \textit{analítica}. Si $k\geq1$ se dice que es una \textbf{variedad diferenciable}.

    
    
%%%%%%%%%%%%%%%%%%%   Vectores y tensores

\subsection{Vectores y tensores}\label{vectores}

Se define una \textit{curva} como un mapa diferenciable de un abierto de $\mathbb{R}$ a $M$. Los puntos en $M$ asociados a los puntos sobre la curva en $\mathbb{R}$ se llaman la \textit{imágen} de la curva. El conjunto de todos los puntos de la imágen corresponde a la noción ordinaria de curva; por esta razón, de ahora en más se usará el término \textit{curva} indistintamente. Consideremos una curva con parámetro $\lambda$ que pasa a través de un punto $P\in M$, descripta por los puntos $x^i=x^i(\lambda)$ en $\mathbb{R}^n$. A su vez, consideremos la función $f:M\rightarrow \mathbb{R}$. La derivada de la función $f$ a lo largo de la curva, evaluada en el punto $P$, se define como 

\begin{align*}
    \frac{df}{d\lambda}\bigg\rvert_P&=\frac{dx^i}{d\lambda}\bigg\rvert_P\frac{\partial f}{\partial x^i} &\Longrightarrow &  &\frac{d}{d\lambda}\bigg\rvert_P=\frac{dx^i}{d\lambda}\bigg\rvert_P\frac{\partial }{\partial x^i}
\end{align*}

lo cual es cierto ya que vale para cualquier función $f$. De esta forma, se define el \textbf{vector tangente} a la curva como un operador de derivación a lo largo de ella (es una derivada direccional). Un vector es un objeto geométrico -independiente de la carta- y $\{dx^i/d\lambda\}$ son las componentes de dicho vector. %Cabe aclarar que cada curva tiene un único vector tangente. 

Se puede probar que las derivadas direccionales a lo largo de las curvas, como es el caso de $d/d\lambda$, satisfacen los axiomas para formar un espacio vectorial. A su vez, se puede observar que cada vector tangente en un punto $P$ se puede escribir como combinación lineal de operadores $\partial/\partial x^i$. Es fácil, pues, ver que  $\{\partial/\partial x^i\}$ forman una base en el espacio vectorial. A dicha base se la conoce como \textbf{base coordenada}, y el espacio vectorial formado por los vectores tangentes al punto $P$ en $M$ se lo conoce como \textbf{espacio vectorial tangente}, notado $T_P(M)$ o simplemente $T_P$, donde dicho espacio posee la misma dimensión que la variedad en cuestión. Cuando a cada punto de la variedad se le puede asignar un vector, diremos que se trata de un \textit{campo vectorial}. 


Ante un cambio de coordenadas $x^i\rightarrow y^i$, las componentes de un vector $V=\frac{d}{d\lambda}=\frac{dx^i}{d\lambda}\frac{\partial}{\partial x^i}=V^i\frac{\partial}{\partial x^i}$ transforman según 

\begin{equation*}
    V^i=\frac{dx^i}{d\lambda}=\frac{dy^j}{d\lambda}\frac{\partial x^i}{\partial y^j}=V^j\frac{\partial x^i}{\partial y^j}
\end{equation*}

El espacio vectorial dual a $T_P$ se lo llama \textbf{espacio cotangente}, y se nota $T^*_P(M)$ o simplemente  $T^*_P$. A sus elementos se los conoce como \textbf{1-formas} y se denotan $dx^i$; a su vez, las 1-formas $\{dx^i\}$ forman una base en $T^*_P$ -tal como sucede con $\{\partial/\partial x^i\}$ en $T_P$- llamada \textit{base dual}.

A un vector se lo suele notar como $\overline{V}$, mientras que a una 1-forma se la nota como $\widetilde{\omega}$. Cuando se aplica un vector a una 1-forma, se define dicha aplicación como $\widetilde{\omega}(\overline{V})\in \mathbb{R}$.

Un \textbf{tensor} del tipo $\dbinom{r}{s}$ en un punto $P$ se define como una función multilineal tal que

\begin{equation*}
T:\underbrace{T^*_P\times ...\times T^*_P}_{r}\times \underbrace{T_P \times ...\times T_P}_{s}\rightarrow\mathbb{R}
\end{equation*}

Los tensores forman un espacio vectorial, cuya dimensión es $n^{r+s}$. Como casos particulares, los vectores son tensores del tipo $\dbinom{1}{0}$ y las 1-formas son tensores del tipo $\dbinom{0}{1}$. A su vez, se define una \textbf{p-forma} como un tensor del tipo $\dbinom{0}{p}$ totalmente antisimétrico. Si tomamos una base $\{v_a\}$ y su base dual $\{v^{*b}\}$, entonces un tensor se puede escribir como

\begin{equation*}
    T=\tensor{T}{^{a_1...a_r}_{b_1...b_s}} v_{a_1}\otimes...\otimes v_{a_r}\otimes v^{*b_1}\otimes...\otimes v^{*b_s}
\end{equation*}

donde $\tensor{T}{^{a_1...a_r}_{b_1...b_s}}$ son las componentes del tensor, las cuales ante un cambio de coordenadas transforman según

\begin{equation*}
    \tensor{T}{^{a'_1...a'_r}_{b'_1...b'_s}}=\tensor{T}{^{a_1...a_r}_{b_1...b_s}}\frac{\partial x'^{a'_1}}{\partial x^{a_1}}...\frac{\partial x^{b_1}}{\partial x'^{b'_1}}...
\end{equation*}

Cuando a cada punto de la variedad se le puede asignar un tensor, diremos que se trata de un \textit{campo tensorial}. 
 

  

%%%%%%%%%%%%%%%%%%%   Tensor Métrico


\subsection{Tensor Métrico} 

Una \textit{métrica Riemanniana} $g$ en una variedad diferenciable es un campo tensorial del tipo $\dbinom{0}{2}$ simétrico ($g(X,Y)=g(Y,X)$) y definido positivo ($g(X,X)>0$, $\forall X\neq 0$). Cuando la métrica es indefinida ($g(X,X)\neq 0$, $\forall X\neq 0$) se dice que es una métrica \textbf{pseudo-Riemanniana}.  Un caso particular de estas últimas son las conocidas como \textbf{métricas Lorentzianas}, las cuales tienen signatura $(-,+,+,+)$ (en dimensión 4). La métrica Lorentziana más simple es la \textit{métrica de Lorentz}, dada por $\eta_{ab}=diag(-1,1,1,1)$, que es la métrica del espacio-tiempo en relatividad especial.

La métrica permite dar una noción de longitud sobre la variedad: se define el intervalo infinetesimal como


\begin{equation*}
    ds^2=g_{ab}dx^adx^b
\end{equation*}
  
Si una curva tiene tangente $V=\frac{d}{d\lambda}$ entonces un elemento de la curva posee longitud

\begin{equation*}
    dl^2=g_{ab}dx^adx^b=g_{ab}\frac{dx^a}{d\lambda}\frac{dx^b}{d\lambda}d\lambda^2=g_{ab}V^aV^bd\lambda^2
\end{equation*}

y por lo tanto

\begin{equation*}
    l=\int\limits_{\lambda_1}^{\lambda_2}\sqrt{\abs{ g_{ab}\frac{dx^a}{d\lambda}\frac{dx^b}{d\lambda}}}d\lambda
\end{equation*}

En este sentido se dice que la métrica permite definir una noción de distancia sobre la variedad. A su vez, la métrica permite definir una noción de ortogonalidad entre vectores: dados dos vectores $X,Y\in T_P$ se dicen que son ortogonales si $g_{ab}X^aY^b=0$. La métrica también da un isomorfismo entre vectores y vectores duales:

\begin{align*}
    \overline{V}\cdot \overline{U}=g(V,U)=g_{ab}V^aU^b& &\Longrightarrow& &V_b=g_{ab}V^a
\end{align*}
    
siendo $V_b$ la componente de una 1-forma ($g(\overline{V},\cdot)=\widetilde{V}$ es una 1-forma). Este isomorfimo entre los espacios tangente y cotangente se lo conoce como \textit{operación de bajar índices}, y la operación inversa se llama \textit{subir índices}. La métrica, con componentes $g_{ab}$, posee una inversa y es lo que permite definir la relación 1-1 entre vectores y vectores duales. La métrica inversa tiene componentes $g^{ab}$ (tal que $g_{ab}g^{bc}=\delta_a^c$)  y se la conoce precisamente como \textit{métrica inversa}.




%%%%%%%%%%%%%%%%%%%  Conexión afín y derivada covariante


\subsection{Conexión afín y derivada covariante}  


%the two differential operators defined by the manifold structure are too limited to serve as the generalization of the concept of a partial derivative one needs in order to set up field equations for physical quantities on the manifold; d operates only on forms, while the ordinary partial derivative is a directional derivative depending only on a direction at the point in question, unlike the Lie derivative. One obtains such a generalized derivative, the covariant derivative, by introducing extra structure on the manifold. 

Dados dos espacios tangentes $V_p$ y $V_q$ en dos puntos distintos $p,q$ de la variedad, no hay manera de saber si un vector en $p$ es el mismo que un vector en $q$, ya que los espacios vectoriales son distintos. En este sentido se dice que dado nada más que la estructura de variedad, no se puede definir naturalmente la noción de transporte paralelo. Por este motivo, la definición de transporte paralelo requiere de algo más además de la estructura de variedad: la \textbf{conexión afín}. Dados dos vectores $U,V$, la \textbf{derivada covariante} del vector $U$ con respecto al vector $V$ es otro vector, $\nabla_VU$, tal que cumple las siguientes propiedades:

\begin{enumerate}[{1)}]
    \item $\nabla_V(\alpha U+\lambda W)=\alpha\nabla_VU+\lambda\nabla_VW$ \quad con $\alpha$ y $\lambda$ escalares
    \item $\nabla_V(fW)=\nabla_V(f)W+f\nabla_V(W)$ (Regla de Leibnitz) \quad donde $\nabla_V(f)=V(f)$ y vale para toda función diferenciable $f$
    \item $\nabla_{(fV+gW)}U=f\nabla_VU+g\nabla_VW$\quad para toda función $f,g$ diferenciables
\end{enumerate}

Sea una base $\{e_a\}$ de $T_p$, el vector $\nabla_VU$ en dicha base será:

\begin{equation*}
    \nabla_VU=\nabla_{V^ae_a}(U^be_b)=V^a[(\nabla_{e_a}U^b)e_b+U^b(\nabla_{e_a}e_b)]=V^a[e_a(U^c)+U^b\Gamma^c_{ba}]e_c
\end{equation*}

donde $\Gamma^c_{ba}$ es la conexión afín, y se la define como $\Gamma^c_{ba}e_c=\nabla_{e_a}e_b$. En una base coordenada (esto es $e_a=\partial/\partial x^a$) a $\Gamma^c_{ba}$ se lo conoce como \textbf{símbolos de Christoffel}, y la derivada covariante en dicha base resulta

\begin{equation*}
    \nabla_VU=V^a[\partial_aU^c+\Gamma^c_{ba}U^b]\frac{\partial}{\partial x^a}
\end{equation*}

Es usual notar a las componentes de la derivada covariante como:

\begin{equation*}
    \nabla_aU^c\equiv U^c_{;a}=U^c_{,a}+\Gamma^c_{ba}U^b
\end{equation*}

donde $\nabla_a=\nabla_{e_a}$ y $\partial_aU^c=U^c_{,a}$.

Se extiende la derivada covariante para un tensor de cualquier tipo como

\begin{equation*}
    \tensor{T}{^{a_1...a_r}_{b_1...b_s;m}}=\tensor{T}{^{a_1...a_r}_{b_1...b_s,m}}+\Gamma^{a_1}_{nm}\tensor{T}{^{na_2...a_r}_{b_1...b_s}}+...-\Gamma^n_{b_1m}\tensor{T}{^{a_1...a_r}_{nb_2...b_s}}-...
\end{equation*}

Sea $\mathcal{C}$ una curva con vector tangente $V$. Un vector tangente $U$ se dice que es transportado paralelamente a lo largo de la curva sii $\nabla_VU=0$.


%%%%%%%%%%%%%%%%%%%   Curvatura 


\subsection{Curvatura}  \label{curvatura}

El \textbf{tensor de torsión} (o simplemente torsión) es un tensor del tipo $\dbinom{1}{2}$ que dado dos vectores $U,V$ se define como:

\begin{equation*}
    T(-;U,V)=\nabla_UV-\nabla_VU-[U,V]
\end{equation*}

En una base coordenada, las componentens del tensor son

\begin{equation*}
    \tensor{T}{^k_{ij}}=\tensor{\Gamma}{^k_{ij}}-\tensor{\Gamma}{^k_{ji}}
\end{equation*}

Si la torsión es nula, se deduce fácilmente que la conexión es simétrica. El \textit{teorema fundamental de geometría Riemanniana} (que vale para geometrías pseudo-Riemannianas también) establece que hay una única conexión con torsión nula tal que preserva la métrica (i.e: $\nabla_ag_{bc}=0$). Esta conexión se llama \textbf{conexión de Levi-Civita}, y los símbolos de Christoffel en este caso vienen dados por

\begin{equation*}
    \tensor{\Gamma}{^l_{jk}}=\frac{1}{2}g^{li}(g_{ij,k}+g_{ik,j}-g_{jk,i})
\end{equation*}

La idea geométrica de la torsión es que cuando el tensor es nulo, es decir cuando la conexión es simétrica, dos geodésicas (siguiente sección) permanecen paralelas en el sentido de que un vector transportado paralelamente permanece ``unido" a la congruencia de geodésicas paralelas. De lo contarario, si la torsión no es nula (conexión no simétrica), dicho vector es ``rotado" en relación a geodésicas cercanas, y por este motivo se dice que la congruencia de geodésicas se ``tuerce".

El \textbf{tensor de Riemann} (o tensor de curvatura) se define como un tensor del tipo $\dbinom{1}{3}$ que dados tres vectores $U,V,W$ devuelve

\begin{equation}\label{definicion Riemann}
    R(-;W,U,V)=[\nabla_U,\nabla_V]W-\nabla_{[U,V]}W
\end{equation}

En base coordenada, las componentes del tensor son

\begin{equation*}
    \tensor{R}{^l_{kij}}=\partial_i\tensor{\Gamma}{^l_{kj}}-\partial_j\tensor{\Gamma}{^l_{ki}}+\tensor{\Gamma}{^m_{kj}}\tensor{\Gamma}{^l_{mi}}-\tensor{\Gamma}{^m_{ki}}\tensor{\Gamma}{^l_{mj}}
\end{equation*}

Geométricamente, el tensor de Riemann habla sobre la diferencia entre un vector tangente y su transporte paralelo a lo largo de una curva cerrada: Si tomamos una cuadrilátero infinetesimal 2-dimensional con coordenadas $s,t$, y un vector $V$, la diferencia entre dicho vector y su transporte paralelo a lo largo del cuadrilátero de lados $\Delta t$, $\Delta s$ resulta

\begin{equation*}
    \delta V^a=\Delta s\Delta t\bigg(\frac{\partial}{\partial t}\bigg)^c\bigg(\frac{\partial}{\partial s}\bigg)^b\tensor{R}{_{cbd}^a}V^d
\end{equation*}

Si el tensor de Riemann se anula, se dice que el transporte paralelo no depende del camino. En un espacio plano (Euclideano), dos rectas paralelas jamás se cruzan: un vector en un punto $p$ se dice paralelo a otro vector en otro punto $q$ porque puedo transportar paralelamente un vector de un punto a otro independientemente del camino. En este sentido, se dice que el tensor de Riemann mide la \textit{curvatura} del espacio, y por lo tanto si el tensor de Riemann es nulo, el espacio se dice plano.

Se definen las siguientes contracciones del tensor de Riemann: \textbf{tensor de Ricci} (notado algunas veces como \textit{Ric}), un tensor del tipo $\dbinom{0}{2}$ dado por la contracción

\begin{equation*}
    R_{ij}=\tensor{R}{^k_{ikj}}
\end{equation*}

y el \textbf{escalar de curvatura} como

\begin{equation*}
    R=g^{ij}R_{ij}
\end{equation*}

A continuación se enuncian ciertas propiedades sobre el tensor de Riemann y sus contracciones:

\begin{lemma}
El tensor de Riemann cumple:
    \begin{enumerate}[{1)}]
        \item $\tensor{R}{^l_{kij}}=-\tensor{R}{^l_{kji}}$
        \item $\tensor{R}{^l_{[kij]}}=0$
        \item $R_{ijkl}=R_{klij}$
        \item Identidad de Bianchi: $\tensor{R}{^l_{k[ij;m]}}=0$
    \end{enumerate}
\end{lemma}

\begin{lemma}
El tensor simétrico de rango 2 más general construído a partir de $R_{ijkl}$, sus contracciones, $g_{ij}$, y simétrico en $R_{ijkl}$ tiene la forma $aR_{ij}+bRg_{ij}+\Lambda g_{ij}$. Posee divergencia nula si $b=-1/2a$ y se anula en un espacio plano si $\Lambda=0$.
\end{lemma}

\begin{lemma}
$R_{ijkl}$, $g_{ij}$ y los tensores construidos a partir de ellos pero lineales en $R_{ijkl}$, son los únicos tensores que se pueden construir con las componentes de $g_{ij}$, $g_{ij,k}$, $g_{ij,kl}$ y que a su vez sean lineales en $g_{ij,kl}$.
\end{lemma}

\begin{lemma}
El tensor de Einstein, $G_{ij}=R_{ij}-\frac{1}{2}Rg_{ij}$ es el único tensor simétrico de rango 2 que cumple las siguientes propiedades:
    \begin{enumerate}[{1)}]
    \item Se puede construir con la componentes de $g_{ij}$, $g_{ij,k}$ y $g_{ij,kl}$
    \item Sus componentes son lineales en $g_{ij,kl}$
    \item Tiene divergencia nula
    \item Se anula si el espacio es plano    
    \end{enumerate}
Si la condición 4) se quita, entonces la forma más general es $G_{ij}+\Lambda g_{ij}$.    
\end{lemma}


%%%%%%%%%%%%%%%%%%%  Geodésicas


\subsection{Geodésicas}

Dado una curva con vector tangente $V$, se dice \textbf{geodésica} si su vector tangente cumple 

\begin{equation*}
    \nabla_VV=\alpha V
\end{equation*}

con $\alpha$ un escalar. Sin embargo, siempre es posible reparametrizar la curva y obtener una ecuación como la del transporte paralelo (llamado también a veces \textit{ecuación geodésica}):

\begin{equation*}
    \nabla_VV=0
\end{equation*}

Una parametrización de este tipo se la conoce como \textit{parametrización afín} y, sin pérdida de generalidad, de ahora en más cuando se habla de geodésicas se considerarán con dicha parametrización. Teniendo esto en cuenta, podremos definir pues a una \textbf{geodésica} como una curva tal que su vector tangente es transportado paralelamente a lo largo de ella.

Sea $U$ el vector tangente a una curva, con $\lambda$ el parámetro afín de la curva; en una base coordenada la ecuación geodésica resulta

\begin{equation*}
    \frac{dU^i}{d\lambda}+\tensor{\Gamma}{^i_{jk}}U^jU^k=0
\end{equation*}

o lo que es lo mismo

\begin{equation*}
    \frac{d^2x^i}{d\lambda^2}+\tensor{\Gamma}{^i_{jk}}\frac{dx^j}{d\lambda}\frac{dx^k}{d\lambda}=0
\end{equation*}

Esta última ecuación es una ecuación diferencial de segundo orden (formalmente, es un sistema de $n$ ecuaciones diferenciales acopladas de segundo orden). Los teoremas de existencia y unicidad de ecuaciones diferenciales establecen que siempre existe una única solución (al menos local) para cualquier condición inicial de $x^i$ y $dx^i/d\lambda$. Esto implica que dado un punto $p\in M$ y un vector tangente $U\in T_p$, siempre existe una única geodésica que pase por $p$ con tangente $U$. 

En términos de geodésicas se define:

\begin{definition}
Dado cualquier punto $a\in M$, se define el \textit{mapa exponencial} como una aplicación suave ($C^\infty$) - denotado como $exp_a$ (donde $exp_a: T_a\rightarrow M$) - que para un conjunto abierto del espacio tangente, devuelve un punto $p\in M$ - si existe - tal que la geodésica (que notaremos como $\lambda$) con vector tangente $V$ en el punto $a$ y que toma el valor $\lambda(a)=0$, obtiene un valor $\lambda(p)=1$ en el punto $p$. Si $V$ tiene componentes ($t,x,y,z$) para alguna base en $T_a$, entonces $t,x,y,z$ se llaman \textit{coordenadas normales de Riemann} en el punto $p$. En dichas coordenadas las derivadas de la métrica se anulan, como así también los símbolos de Christoffel.
\end{definition}

\begin{definition}
Dado un entorno $N_p$ de un punto $p\in M$, se dice \textit{entorno convexo (normal)} si cualquier punto $q\in N_p$ puede ser unido con otro punto $r\in N_p$ por una única geodésica que comienza en $q$ y está totalmente contenida en $N_p$.
\end{definition}
    
    

      
   
%%%%%%%%%%%%%%%%%%% Relatividad General


\section{Relatividad General}

En esta sección se repasan brevemente los conceptos, postulados y ecuaciones de la Relatividad General. Se sigue como referencias \citep{1984ucp..book.....W,1975lsss.book.....H,2009fcgr.book.....S} 

%%%%%%%%%%%%%%%%%%%   Espacio-tiempo y postulados


\subsection{Espacio-tiempo y postulados}

El primer concepto que surge de interés a la hora de estudiar Relatividad General es el de \textit{espacio-tiempo}:

\begin{definition}
$(M,g_{ab})$ es un \textbf{espacio-tiempo} si $M$ es una variedad diferenciable con una métrica lorentziana definida, $g_{ab}$.
\end{definition}


Se agrega a la definición de espacio-tiempo que $M$ sea una variedad real 4-dimensional, $C^\infty$, Hausdorff y conexa, y que la conexión afín sea la de Levi-Civita. Además, definido así, el espacio-tiempo resulta paracompacto.

Un vector se puede clasificar de tres formas distintas según su norma:

\begin{definition}
Sea $(M,g_{ab})$ un espacio-tiempo y un punto $p\in M$. Un vector tangente $X\in T_p$ se dice:
    \begin{itemize}
        \item Temporal (\textit{Timelike}) \qquad si  $g_{ab}X^aX^b<0$
        \item Espacial (\textit{Spacelike}) \hspace{0.8cm} si  $g_{ab}X^aX^b>0$
        \item Nulo      \hspace{3.3cm} si  $g_{ab}X^aX^b=0$
    \end{itemize}
Un vector se dice que es \textit{causal} si es temporal o nulo.
\end{definition}

Como consecuencia del uso de la conexión de Levi-Civita, se puede probar que las partículas libremente gravitantes se mueven a lo largo de geodésicas temporales de la métrica y que la luz se mueve a lo largo de geodésicas nulas, en concordancia con lo que sucede en Relatividad Especial. Este resultado se lo conoce como \textit{Principio de Equivalencia Débil}. 

A continuación se enuncian los postulados que gobiernan la Relatividad General: el primer postulado es el conocido como el de \textit{causalidad local} y establece que las ecuaciones que gobiernan los campos de materia deben ser tales que si $U\subseteq M$ es un entorno convexo y $p,q\in U$, entonces una señal puede ser enviada entre los puntos $p$ y $q$ sii $p$ y $q$ pueden ser unidos mediante una curva causal $C^1$ (por lo menos), tal que esté totalmente contenida en $U$ y que su vector tangente sea no nulo en todos lados. 

El siguiente postulado se lo conoce como el de \textit{conservación local de energía y momento}; el mismo establece que las ecuaciones que gobiernan los campos de materia son tales que existe un tensor simétrico $T^{ab}$ -llamado \textit{tensor de energía-momento}- que depende del campo, su derivada covariante y de la métrica, y que posee las siguientes propiedades:

\begin{enumerate}[{(1)}]
    \item $T^{ab}$ se anula en un abierto $U\subseteq M$ sii todos los campos de materia se anulan en $U$
    \item $T^{ab}$ satisface la ecuación de conservación $\nabla_aT^{ab}=0$
\end{enumerate}

El último postulado relaciona el tensor de energía-momento junto con la geometría del espacio-tiempo: en todo el espacio-tiempo, $(M,g_{ab})$, se cumplen las \textbf{ecuaciones de Einstein}

\begin{equation*}
    R_{ab}-\frac{1}{2}Rg_{ab}=kT_{ab}
\end{equation*}

donde $k=8\pi G/c^4$. 

Habiendo desarrollado las ideas previas, se podrían resumir los contenidos de Relativida General en la siguiente frase: \textit{El espacio-tiempo en Relatividad General es una variedad 4-dimensional con una métrica lorentziana definida en ella, en donde la curvatura del espacio-tiempo está relacionada con la distribución de materia vía las ecuaciones de Einstein}.





%%%%%%%%%%%%%%%%%%%%   Formulación Lagrangiana - Ecuaciones de Einstein


\subsection{Formulación Lagrangiana - Ecuaciones de Einstein}\label{EH action}

Las ecuaciones de Einstein pueden ser deducidas a partir de una acción conocida como la \textit{acción de Hilbert-Einstein}:

\begin{equation*}
    S=\frac{1}{2k}\int d^4x\sqrt{-g}R
\end{equation*}

donde $g=\text{det}(g_{ij})$ y $k=8\pi G/c^4$. Variaremos la acción, pidiendo que se extreme, tomando como variable dinámica a la métrica:

\begin{equation*}
    0=\delta S=\frac{1}{2k}\int d^4x[\delta R\sqrt{-g}+R\delta\sqrt{-g}]
\end{equation*}

La variación del escalar de curvatura viene dada por

\begin{equation*}
    \delta R=\delta (g^{ij}R_{ij})=\delta g^{ij} R_{ij}+g^{ij}\delta R_{ij}
\end{equation*}

y por lo tanto la variación de la acción resulta 

\begin{equation*}
    \delta S=\frac{1}{2k}\int d^4x\left[(\delta g^{ij} R_{ij}+g^{ij}\delta R_{ij})\sqrt{-g}+R\left(-\frac{1}{2}\frac{1}{\sqrt{-g}}\delta g   \right)   \right]
\end{equation*}

La \textit{regla de Jacobi} establece que 

\begin{equation*}
    \delta g \equiv \delta (\text{det}g)=gg^{ij}\delta g_{ij}
\end{equation*}

Teniendo en cuenta que $g^{ij}\delta g_{ij}=g^{ij}\delta (g_{ik}g_{lj}g^{kl})$ y que $g^{ij}g_{jk}=\delta ^i_k$, se puede probar fácilmente que

\begin{equation*}
    \delta g=gg^{ij}\delta g_{ij}=-gg_{ij}\delta g^{ij}
\end{equation*}

lo cual reemplazando en la acción se obtiene que

\begin{equation*}
    \delta S=\frac{1}{2k}\int d^4x \sqrt{-g}\left[ \left(R_{ij}-\frac{1}{2}Rg_{ij}\right)\delta g^{ij}+g^{ij}\delta R_{ij} \right]
\end{equation*}

Veamos que el término con la variación del Ricci, $g^{ij}\delta R_{ij}$, se anula: sabiendo que el tensor de Riemann se escribe en término de la conexión como 

\begin{equation*}
    \tensor{R}{^l_{kij}}=\partial_i\tensor{\Gamma}{^l_{kj}}-\partial_j\tensor{\Gamma}{^l_{ki}}+\tensor{\Gamma}{^m_{kj}}\tensor{\Gamma}{^l_{mi}}-\tensor{\Gamma}{^m_{ki}}\tensor{\Gamma}{^l_{mj}}
\end{equation*}

la variación del mismo da

\begin{equation*}
    \delta \tensor{R}{^l_{kij}}=\partial_i\delta\tensor{\Gamma}{^l_{kj}}-\partial_j\delta\tensor{\Gamma}{^l_{ki}}+\delta\tensor{\Gamma}{^m_{kj}}\tensor{\Gamma}{^l_{mi}}+\tensor{\Gamma}{^m_{kj}}\delta\tensor{\Gamma}{^l_{mi}}-\delta\tensor{\Gamma}{^m_{ki}}\tensor{\Gamma}{^l_{mj}}-\tensor{\Gamma}{^m_{ki}}\delta\tensor{\Gamma}{^l_{mj}}
\end{equation*}

A pesar que la conexión no es un tensor, la diferencia entre dos de ellas sí es un tensor \citep{Lovelock-Rund}, y por lo tanto se puede calcular su derivada covariante:

\begin{equation*}
    \nabla_l(\delta\tensor{\Gamma}{^k_{ij}})=\partial_l(\delta\tensor{\Gamma}{^k_{ij}})+\tensor{\Gamma}{^k_{ml}}\delta\tensor{\Gamma}{^m_{ij}}-\tensor{\Gamma}{^m_{li}}\delta\tensor{\Gamma}{^k_{mj}}-\tensor{\Gamma}{^m_{jl}}\delta\tensor{\Gamma}{^k_{im}}
\end{equation*}

De esta forma, la variación del tensor de Riemann resulta el términos de la derivada covariante de la conexión como

\begin{equation*}
    \delta \tensor{R}{^l_{kij}}=\nabla_i(\delta\tensor{\Gamma}{^l_{kj}})-\nabla_j(\delta\tensor{\Gamma}{^l_{ik}})
\end{equation*}

lo cual es evidente si se expande la derivada covariante de las conexiones, y se tiene en cuenta que $\tensor{\Gamma}{^k_{ij}}$ es simétrica en los dos índices inferiores. Hay que notar que cada derivada covariante aporta cuatro términos, pero usando que los índices son ``\textit{mudos}", dos de ellos se cancelan y se obtiene exactamente la ecuación para la variación del Riemann. 

De esta forma se llega a una ecuación para la variación del Ricci, la cual viene dada por

\begin{equation*}
    \delta\tensor{R}{^k_{ikj}}=\delta R_{ij}=\nabla_k(\delta\tensor{\Gamma}{^k_{ji}})-\nabla_j(\delta\tensor{\Gamma}{^k_{ki}})
\end{equation*}

que es conocida también como \textit{identidad de Palatini}.

Mediante algunos cálculos se puede demostrar que la variación de la conexión resulta

\begin{equation*}
    \delta\tensor{\Gamma}{^k_{ij}}=-\frac{1}{2}\left[ \nabla_i(g_{jm}\delta g^{km})+\nabla_j(g_{im}\delta g^{km})-\nabla_n(g_{il}g_{jm}g^{kn}\delta g^{lm}) \right]
\end{equation*}

y teniendo esto en cuenta se puede probar que

\begin{equation*}
    g^{ij}\delta R_{ij}=\nabla_i\nabla_j(-\delta g^{ij}+g^{ij}g_{km}\delta g^{km})
\end{equation*}

Si llamamos $V^i=\nabla_j(-\delta g^{ij}+g^{ij}g_{km}\delta g^{km})$ entonces el término con la variación del Ricci en la acción resulta

\begin{equation*}
	\int d^4x\sqrt{-g}g^{ij}\delta R_{ij}=\int d^4x\sqrt{-g} (\nabla_iV^i)
\end{equation*}

y como para cualquier vector vale que 

\begin{equation*}
	\sqrt{-g}(\nabla_iV^i)=\partial_i(\sqrt{-g}V^i)
\end{equation*}

el término con la variación del Ricci resulta una derivada total y, en consecuencia, por el Teorema de Stokes se anula. Luego, pedir que $\delta S=0$ equivale a pedir que se cumplan las ecuaciones de Einstein en vacío:

\begin{equation*}
	G_{ij}=R_{ij}-\frac{1}{2}Rg_{ij}=0
\end{equation*}

Por último, si se modifica la acción previa se pueden llegar a diversos resultados conocidos. Si por ejemplo se desea agregar el término de materia a la acción, la misma resulta

\begin{equation*}
	S'=S+\int d^4x\sqrt{-g}\mathcal{L}_{mat}
\end{equation*}

y la misma dará las ecuaciones de Einstein con materia, $R_{ij}-\frac{1}{2}Rg_{ij}=kT_{ij}$, en donde el tensor de energía-momento viene dado por

\begin{equation*}
	T_{ij}=\frac{-2}{\sqrt{-g}}\frac{\delta (\sqrt{-g}\mathcal{L}_{mat})}{\delta g^{ij}}
\end{equation*}

Notemos que si tomamos traza en la ecuación de Einstein, se obtiene que $R=-kT$. A su vez, si se desea agregar el término con constante cosmológica, $\Lambda$, se puede hacer mediante

\begin{equation*}
	S=\frac{1}{2k}\int d^4x\sqrt{-g}(R-2\Lambda)
\end{equation*}

En este caso se obtienen las ecuaciones de Einstein con constante cosmológica: $R_{ij}-\frac{1}{2}Rg_{ij}+\Lambda g_{ij}=0$.

Si se combinan ambos resultados se obtiene uno más general:

\begin{align*}
    	S=\frac{1}{2k}\int d^4x\sqrt{-g}(R-2\Lambda+\mathcal{L}_{mat})& &\Longrightarrow& & R_{ij}-\frac{1}{2}Rg_{ij}+\Lambda g_{ij}=kT_{ij}
\end{align*}

que son las ecuaciones de Einstein con constante cosmológica en presencia de materia.


