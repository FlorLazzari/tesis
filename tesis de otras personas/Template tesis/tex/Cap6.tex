\chapter{Gravedad de Gauss-Bonnet}\label{GB}

En el capítulo anterior hemos visto una generalización de los Teoremas de Singularidad de Hawking aplicado a un modelo inflacionario de Higgs. En dicho modelo vimos que no se cumplía la SEC, la cual es una hipótesis fundamental en la formulación de los teoremas originales, pero que sin embargo podíamos aplicar los teoremas dado que podíamos imponer ciertas restricciones sobre el tensor de Ricci debido a que el potencial se hallaba acotado. La aplicación de dichos teoremas en un contexto de gravedad de Gauss-Bonnet, en cambio, resultan más complejas que el caso del modelo inflacionario de Higgs. Esto se debe a que las ecuaciones de movimiento, en este caso, involucran no solo al tensor de Ricci, sino también a la curvatura y, por lo tanto, resulta difícil expresar al tensor de Ricci en términos del tensor de energía-momento (como veremos en la sección \ref{ec mov GB}). Por ende, resulta así muy complicado derivar consecuencias físicas a partir de la ecuación de Raychaudhuri, y las técnias presentadas en el capítulo anterior sobre los Teoremas de Singularidades se ven limitados. Sin embargo, presentaremos un avance sobre el tema en este capítulo considerando un campo escalar acoplado al término de Gauss-Bonnet y, veremos que bajo ciertas situaciones, se pueden estudiar singularidades en la teoría, como así también un estudio sobre la evolución del universo. Las referencias a seguir son \citep{2017arXiv170301713S,2015arXiv151200222H}






%%%%%%%%%   Acción y Ecuaciones de Movimiento

    
\section{Acción y Ecuaciones de Movimiento}\label{ec mov GB}


En la sección \ref{EH action} hemos estudiado la acción de Hilbert-Einstein. El modelo de Gauss-Bonnet consiste en modificar dicha acción agregando términos invariantes de la curvatura en la misma, que se puede escribir de la siguiente manera en $D$ dimensiones

\begin{equation}
    S=\int d^Dx \sqrt{-g}G
\end{equation}


donde el término de Gauss-Bonnet $G$ viene dado por 

$$
G=R^2-4R_{ab}R^{ab}+R_{abcd}R^{abcd}
$$


Las ecuaciones de movimiento resultantes de variar la métrica $\delta g_{ab}$ son

$$
-\frac{1}{2}R_{cd}R^{cd}g_{ab} - \nabla_a\nabla_bR - 2R_{cbad}R^{cd} + \frac{1}{2}g_{ab}\Box R + \Box R_{ab} + \frac{1}{2}R^{2}g_{ab} - 2RR_{ab}
$$
\begin{equation}    
-2\nabla_b\nabla_aR + 2g_{ab}\Box R -\frac{1}{2}Rg_{ab} + R_{ab}=0.
\end{equation}

Si $D=4$ entonces el término $\sqrt{-g}G$ puede ser escrito como una derivada total

\begin{equation*}
    \sqrt{-g}G=\partial_aK^a ,
    \qquad
    K^a=\sqrt{-g}\epsilon^{abcd}\tensor{\epsilon}{_{ij}^{kl}}\tensor{\Gamma}{^{i}_{kb}}\bigg[\frac{\tensor{R}{^{j}_{lcd}}}{2}+\frac{\tensor{\Gamma}{^{j}_{mc}}\tensor{\Gamma}{^{m}_{ld}}}{3}\bigg]
\end{equation*}
    
    
    
y por lo tanto en 4 dimensiones este modelo es trivial (y en general, para dimensiones $D\leq 4$). Sin embargo, podemos considerar la siguiente acción modificada, donde el término de Gauss-Bonnet se acopla a un campo escalar $\phi$:

\begin{equation*}
S=\int d^4x \sqrt{-g}\left\{ \frac{1}{2\kappa^2}R - \frac{1}{2}\partial_a\phi \partial^a\phi + V(\phi) + f(\phi)G\right\}
\end{equation*} 


donde ahora debido a este acoplamiento, el lagrangiano ya no resulta una derivada total y por lo tanto las ecuaciones de movimiento adquieren dinámica. Debido al acoplamiento del campo escalar, muchas veces se suele llamar a esta acción también como \textit{acción de Einstein-scalar-Gauss-Bonnet}, que es como muchas veces suele aparecer en la bibliografía.

La ecuación de movimiento para el campo escalar $\phi$ resulta

\begin{equation}\label{ec mov phi}
\nabla^2\phi + f'(\phi)G + V'(\phi)=0
\end{equation}

mientras que las ecuaciones de movimiento para la métrica $g_{ab}$ resultan 

$$
0=\frac{1}{\kappa^2}\left(-R^{ab} + \frac{1}{2}g^{ab}R\right) + \frac{1}{2}\partial^a\phi \partial^b\phi - \frac{1}{4}g^{ab}\partial_c\phi \partial^c\phi + \frac{1}{2}g^{ab}(V(\phi)+f(\phi)G) - 2f(\phi)RR^{ab} 
$$
$$
+2\nabla^a\nabla^b\left(f(\phi)R\right)- 2g^{ab}\nabla^2\left(f(\phi)R\right) + 8f(\phi)\tensor{R}{^a_c} R^{bc} - 4\nabla_c\nabla^a(f(\phi)R^{bc})- 4\nabla_c\nabla^b \left(f(\phi)R^{ac}\right)
$$
$$
+4\nabla^2(f(\phi)R^{ab})+ 4g^{ab}\nabla_c\nabla_d(f(\phi)R^{cd}) - 2f(\phi)R^{aclm}\tensor{R}{^b_{clm}} + 4\nabla_c\nabla_d(f(\phi) R^{acdb}).
$$

Sin embargo, teniendo en cuenta las siguientes identidades consecuencia de las Identidades de Bianchi

$$
\nabla^cR_{cdab}=\nabla_aR_{bd} - \nabla_bR_{ac} ,
$$
$$
\nabla^cR_{ca} = \frac{1}{2}\nabla_aR ,
$$
$$
\nabla_c\nabla_dR^{acbd} = \nabla^2R^{ab} - \frac{1}{2}\nabla^a\nabla^bR
+ R^{acbd} R_{cd} - \tensor{R}{^a_c}R^{bc} ,
$$
$$      
\nabla_c\nabla^aR^{cb} + \nabla_c\nabla^bR^{ca} = \frac{1}{2}(\nabla^a \nabla^bR + \nabla^b\nabla^aR) - 2R^{acbd} R_{cd} + 2\tensor{R}{^a_c} R^{bc},
$$
$$
\nabla_c\nabla_dR^{cd} = \frac{1}{2}\Box R ,
$$

entonces las ecuaciones de movimiento para la métrica se pueden escribir como 

$$
0=\frac{1}{\kappa^2}\left(-R^{ab} + \frac{1}{2}g^{ab}R\right) +  \left(\frac{1}{2}\partial^a\phi \partial^b\phi - \frac{1}{4}g^{ab} \partial_c\phi \partial^c\phi \right) + \frac{1}{2}g^{ab}(V(\phi)+f(\phi)G)
$$   
$$  
-2f(\phi)RR^{ab} + 4f(\phi)\tensor{R}{^a_{c}}R^{bc} -2f(\phi)R^{aclm}\tensor{R}{^b_{clm}} + 4f(\phi)R^{acdb}R_{cd} 
$$
$$ 
+ 2(\nabla^a\nabla^bf(\phi))R - 2g^{ab}(\nabla^2f(\phi))R - 4(\nabla_c\nabla^af(\phi))R^{bc} - 4(\nabla_c\nabla^bf(\phi))R^{ac} 
$$
\begin{equation}\label{ec mov gab}
+ 4(\nabla^2f(\phi))R^{ab} + 4g^{ab}(\nabla_c\nabla_df(\phi))R^{cd} - 4(\nabla_c\nabla_df(\phi))R^{acbd}
\end{equation}

en donde se puede ver que, como dijimos al principio del capítulo, obtener una relación entre el tensor de Ricci y el tensor de energía-momento resulta difícil de hallar. Las ecuaciones (\ref{ec mov phi}) y (\ref{ec mov gab}) son el sistema de ecuaciones que describen completamente la teoría.

Para un universo isótropo y homogéneo, con curvatura espacial cero, la métrica viene dada por

$$
ds^2=-dt^2 + a^2(t)\sum\limits_{i=1}^3 dx_i^2
$$

donde la conexión y las componentes del tensor de Rieman y del Ricci (no nulas) vienen dados por

$$
\Gamma^t_{ij}=a^2 H \delta_{ij}\ ,\qquad
\Gamma^i_{jt}=\Gamma^i_{tj}=H\delta^i_{\ j}\ ,
\qquad R_{itjt}=-\left(\dot H + H^2\right)\delta_{ij},
$$
$$
R_{ijkl}=a^4 H^2\left(\delta_{ik} \delta_{lj} - \delta_{il} \delta_{kj}\right),\qquad
R_{tt}=-3\left(\dot H + H^2\right)\ ,
$$
\begin{equation}\label{wk}
R_{ij}= a^2 \left(\dot H
+ 3H^2\right)\delta_{ij},\qquad R= 6\dot H + 12 H^2
\end{equation}


Mediante el uso de estas fórmulas, la ecuación de movimiento para el campo escalar (\ref{ec mov phi}) se puede escribir como 


\begin{equation}\label{ec mov phi 2}
\ddot{\phi}+3H\dot{\phi}-24H^2 f'(\phi)(H^2+\dot{H})+V'(\phi)=0
\end{equation}


mientras que de las ecuaciones de movimiento para la métrica (\ref{ec mov gab}) se desprenden dos ecuaciones independientes a saber

\begin{equation}\label{ec mov gab 2}
H^2=\frac{\kappa^2}{3}\rho_{eff},\qquad 2\dot{H}+3H^2=-\kappa^2 p_{eff}
\end{equation}

donde la densidad de energía y la presión vienen dadas respectivamente por

$$
\rho_{eff}=\frac{\dot{\phi}^2}{2}+V(\phi)-24 H^3 \dot{f}
$$
$$
p_{eff}=\frac{\dot{\phi}^2}{2}-V(\phi)+8H^2 f''(\phi)\dot{\phi}^2+8H^2 f'(\phi)\ddot{\phi}+16 H\dot{H} f'(\phi)\dot{\phi}+16 H^3 f'(\phi)\dot{\phi}
$$


Las ecuaciones (\ref{ec mov phi 2}) y (\ref{ec mov gab 2}) caracterizan la evolución del universo isótropo y homogéneo con curvatura espacial nula.








%%%%%%%%%   Comportamiento del parámetro $H$ 

    
\section{Comportamiento del parámetro $H$} 


En esta sección estudiaremos las posibles soluciones a las ecuaciones (\ref{ec mov phi 2}) y (\ref{ec mov gab 2}). Antes de comenzar, redefiniremos $8f(\phi)\rightarrow f(\phi)$ y pondremos la constante de Newton $\kappa=1$ para simplificar los cálculos. Luego de estas redefiniciones, las ecuaciones (\ref{ec mov phi 2}) y (\ref{ec mov gab 2}) se pueden escribir como 

\begin{equation}\label{1}
\frac{\dot{\phi}^2}{2}+V(\phi)=3H^2(1+\dot{f}(\phi)H)
\end{equation}


\begin{equation}\label{2}
\frac{\dot{\phi}^2}{2}-V(\phi)=-2(H^2+\dot{H})(1+\dot{f}(\phi)H)-H^2(1+\ddot{f}(\phi))
\end{equation}


\begin{equation}\label{3}
\ddot{\phi}=-3H\dot{\phi}+3H^2 f'(\phi)(H^2+\dot{H})-V'(\phi)
\end{equation}


en donde se puede observar que las mismas son un sistema de ecuaciones no lineales para el campo $\phi$ y la constante de Hubble $H$. Sin embargo, a partir de dichas ecuaciones, se pueden obtener diversas conclusiones sin necesidad de resolver explícitamente el sistema.

Teniendo en cuenta que $\dot{f}(\phi)=f'(\phi)\dot{\phi}$, la ecuación (\ref{1}) se convierte en una ecuación cuadrática para $\dot{\phi}$, cuya solución es

\begin{equation}\label{1 bis}
\dot{\phi}=3H^3f'(\phi)\pm \sqrt{9H^6f'(\phi)^2+6H^2-2V(\phi)}
\end{equation}

Si reemplazamos esta última ecuación en (\ref{3}), obtenemos una expresión para $\ddot{\phi}$ a saber

\begin{equation}\label{phi ddot 1}
\ddot{\phi}=-3H\bigg[3H^3f'(\phi)\pm \sqrt{9H^6f'(\phi)^2+6H^2-2V(\phi)}\bigg]+3H^2 f'(\phi)(H^2+\dot{H})-V'(\phi)
\end{equation}


A su vez, podemos llegar a otra expresión para $\ddot{\phi}$ derivando (\ref{1 bis}). Primero, notemos que $\frac{\partial V(\phi)}{\partial t}=\dot{\phi}V'(\phi)$ y que $\frac{\partial f'(\phi)}{\partial t}=f''(\phi)\dot{\phi}$. Derivando dicha ecuación, obtenemos que


$$
\ddot{\phi}=\dot{H}\bigg[9H^2 f'(\phi)^2\pm \frac{54H^5 f'(\phi)^2+12H}{\sqrt{9H^6f'(\phi)^2+6H^2-2V(\phi)}}\bigg]
$$
\begin{equation}\label{phi ddot 2}
+\bigg[3H^3f'(\phi)\pm \sqrt{9H^6f'(\phi)^2+6H^2-2V(\phi)}\bigg]\bigg[3H^3 f''(\phi)\pm \frac{18H^6 f'(\phi)f''(\phi)-2V'(\phi)}{\sqrt{9H^6f'(\phi)^2+6H^2-2V(\phi)}}\bigg]
\end{equation}



Las ecuaciones (\ref{phi ddot 1}) y (\ref{phi ddot 2}) deben ser iguales ya que ambas representan ecuaciones para $\ddot{\phi}=\ddot{\phi}(H,\dot{H},\phi)$. Esta igualdad implica que 

$$
\bigg[6H^2f'(\phi)\pm \frac{54H^5 f'(\phi)^2+12H}{\sqrt{9H^6f'(\phi)^2+6H^2-2V(\phi)}}\bigg]\dot{H}=3H^4f'(\phi)-\bigg[3H^3f'(\phi)+
$$
\begin{equation}\label{H dot 1}
\pm \sqrt{9H^6f'(\phi)^2+6H^2-2V(\phi)}\bigg]\times\bigg[3H+3H^3f''('\phi)\pm\frac{18H^6 f'(\phi)f''(\phi)-2V'(\phi)}{\sqrt{9H^6f'(\phi)^2+6H^2-2V(\phi)}}\bigg]-V'(\phi)
\end{equation}


Esta última ecuación expresa la evolución del parámetro de Hubble con respecto al tiempo, $\dot{H}$, como función de $\dot{H}=\dot{H}(\phi,H)$. Notemos que para arribar a esta relación, la ecuación (\ref{2}) no ha sido tenida en cuenta. Sin embargo, mediante el uso de esta última, se puede obtener una ecuación de $\dot{H}=\dot{H}(\phi,H)$ no equivalente a (\ref{H dot 1}). Teniendo en cuenta que $\ddot{f}=f''(\phi)\dot{\phi}^2+f'(\phi)\ddot{\phi}$, junto con las ecuaciones (\ref{1}) y (\ref{1 bis}), entonces la ecuación (\ref{2}) puede ser expresada de la siguiente manera:

\begin{equation}\label{2 bis}
-H^2f'(\phi)\ddot{\phi}=H^2+\bigg(1+Hf'(\phi)\dot{\phi}\bigg)\bigg(5H^2+6H^4f''(\phi)+2\dot{H}\bigg)-2V(\phi)\bigg(1+H^2f''(\phi)\bigg)
\end{equation}


Reemplazando $\ddot{\phi}$ por (\ref{phi ddot 1}) y $\dot{\phi}$ por (\ref{1 bis}) en esta última expresión (\ref{2 bis}), obtenemos otra expresión para  $\dot{H}=\dot{H}(\phi,H)$:

$$
\bigg\{2+Hf'(\phi)\bigg[9H^3f'(\phi) \pm 2\sqrt{9H^6f'(\phi)^2+6H^2-2V(\phi)}\bigg]\bigg\}\dot{H}=-H^2
$$
$$
-H^2f'(\phi)\bigg\{-3H\bigg[3H^3f'(\phi) \pm \sqrt{9H^6f'(\phi)^2+6H^2-2V(\phi)}\bigg]+3H^4f'(\phi)-V'(\phi)\bigg\}
$$
\begin{equation}\label{H dot 2}
-\bigg\{1+Hf'(\phi)\bigg[3H^3f'(\phi) \pm \sqrt{9H^6f'(\phi)^2+6H^2-2V(\phi)}\bigg]\bigg\}(5H^2+6H^4 f''(\phi))+2V(\phi)(1+H^2f''(\phi))
\end{equation}

Las ecuaciones (\ref{H dot 1}) y (\ref{H dot 2}) son ecuaciones que describen $\dot{H}=\dot{H}(\phi,H)$. El punto crucial es que ambas ecuaciones son de la forma

$$
A\dot{H}=B \qquad C\dot{H}=D
$$

donde $A,B,C,D$ son funciones de $H$ y $\phi$. La condición para que ambas ecuaciones sean compatibles es que $AD=BC$, lo que se traduce en la siguiente condición mediante algunos cálculos elementales



$$
\bigg[81H^{13}f'(\phi)^4f''(\phi)+243H^{11}f'(\phi)^4+108H^9f'(\phi)^2f''(\phi)+H^7f'(\phi)^2 \times
$$
$$
\times  \bigg(216-108f''(\phi)V'(\phi)+18f'(\phi)V'(\phi)+90f''(\phi)V(\phi)\bigg)+H^5\bigg(36f''(\phi)-54f'(\phi)^2V(\phi)\bigg)+
$$
$$
+H^3\bigg(36-24f''(\phi)V'(\phi)+12f'(\phi)V'(\phi)+12f''(\phi)V(\phi)\bigg)-H\bigg(12V(\phi)+4f'(\phi)V(\phi)V'(\phi)\bigg)\bigg]^2=
$$
$$
\bigg(9H^6f'(\phi)^2+6H^2-2V(\phi)\bigg)\times\bigg[27H^{10}f'(\phi)^3f''(\phi)+54H^8f'(\phi)^3+18H^6f'(\phi)f''(\phi)+
$$
\begin{equation}\label{H poli}
+3H^4f'(\phi)\bigg(4+5f'(\phi)V'(\phi)-4f''(\phi)V'(\phi)+4f''(\phi)V(\phi)\bigg)-12H^2f'(\phi)V(\phi)+2V'(\phi)\bigg]^2
\end{equation}


Esta útima expresión otorga una relación implícita $H=H(\phi)$. De la ecuación (\ref{1 bis}) podemos obtener una ecuación para $\phi=\phi(t)$ a saber


\begin{equation}\label{tiempo}
t-t_0=\int_{\phi_0}^{\phi} \frac{d\phi}{3H^3f'(\phi)\pm \sqrt{9H^6f'(\phi)^2+6H^2-2V(\phi)}}
\end{equation}


Las ecuaciones (\ref{H poli}) y (\ref{tiempo}) representan las soluciones a las Ec. de Einstein en el caso isótropo y homogéneo. Respectivamente otorgan $H=H(\phi)$ y $\phi=\phi(t)$ y, por ende, se puede obtener una solución $H=H(t)$. Sin embargo, como se puede observar, una ecuación explícita para $H(t)$ puede resultar complicada de hallar. A pesar de eso, veremos a continuación que se pueden sacar ciertas conclusiones sin la necesidad de cálculos explícitos sobre ellos.

Para fijar conceptos consideremos el caso en el que no hay potencial $V(\phi)$. En ese caso, la ecuación (\ref{H poli}) se escribe como

$$
\bigg[81H^{10}f'(\phi)^4f''(\phi)+243H^8f'(\phi)^4+108H^6f'(\phi)^2f''(\phi)+216H^4f'(\phi)^2 +36H^2f''(\phi)+36\bigg]^2H^6=
$$
\begin{equation}\label{compo}
    \bigg(9H^6f'(\phi)^2+6H^2\bigg)\times\bigg[27H^6f'(\phi)^3f''(\phi)+54H^4f'(\phi)^3+18H^2f'(\phi)f''(\phi)+12f'(\phi)\bigg]^2H^8
\end{equation}


De este último resultado se puede ver que $H=0$ es una posible solución del sistema. Más aún, siguiendo el caso en el que $V(\phi)=0$, las ecuaciones (\ref{H dot 1}) y (\ref{H dot 2}) muestran que (en ambos casos) $\dot{H}\to 0$ cuando $H\to 0$ (suponiendo que $f(\phi)$ y sus derivadas no son divergentes). Es decir, el espacio plano es una posible solución de este modelo. El caso en donde sí hay $V(\phi)$ se traduce en que el espacio plano deja de ser una posible solución a dicho sistema, ya que en ese caso no necesariamente $H=0$ es una solución. De la ecuación (\ref{H poli}) se tiene que si $H=0$ entonces $0=V(\phi)[V'(\phi)]^2$. De aquí podemos ver que efectivamente si $V(\phi)=0$ entonces se satisface la igualdad teniendo así como resultado un espacio plano. Sin embargo, vemos que para que la ecuación valga (admitiendo como posible solución un espacio plano) el potencial debe tener algún mínimo y/o máximo de forma tal que la derivada se anule. Ciertas condiciones sobre el potencial serán estudiadas más adelante, teniendo en cuenta el resultado recién mencionado. En otras palabras, podemos pensar que la acción del potencial es impedir que el parámetro de Hubble pueda alcanzar un valor nulo, quedando siempre un valor remanente del mismo. Más aún, veremos en la siguiente sección que el potencial no solo puede impedir que el espacio plano sea solución del sistema, sino que también puede impedir que se forme una singularidad en un tiempo inicial $t=0$.



%%%%%%%%%   Cotas para la evolución 

    
\section{Cotas para la evolución} 

De la ecuación (\ref{1}) se deduce inmediatamente que

\begin{equation}\label{V menor}
1+\dot{f}H\geq \frac{V(\phi)}{3H^2}
\end{equation}

Por otro lado - y usando la ecuación (\ref{1}) - podemos deducir a partir de la ecuación (\ref{2}) que 

\begin{equation}\label{V mayor}
\frac{d}{dt}(2H+\dot{f}H^2) \leq \frac{1}{3}V(\phi)
\end{equation}

A su vez, de la ecuación (\ref{1 bis}), se deduce otra condición para el potencial pidiendo que el argumento de la raíz sea no negativo

\begin{equation}\label{V menor 2}
9H^6f'(\phi)^2+6H^2 \geq 2V(\phi)
\end{equation}




Por otro lado, de la ecuación (\ref{1 bis}) obtenemos que 

$$
\dot{\phi}=3H^3f'(\phi)-\sqrt{9H^6f'(\phi)^2+6H^2-2V(\phi)}
$$
\begin{equation}\label{rama negativa}
=3H^3f'(\phi)\left[ 1-\sqrt{1+\frac{6H^2-2V(\phi)}{9H^6f'(\phi)^2}}\right]
\end{equation}

y se siguen ciertos resultados para $\dot{\phi}$:

\begin{itemize}
    \item $\underline{f'(\phi)\to \pm \infty} \qquad \Longrightarrow \qquad\dot{\phi}\simeq -\frac{3H^2-V(\phi)}{3H^3f'(\phi)}\longrightarrow 0$
    \item $\underline{H\to \pm \infty, f'(\phi)\neq 0} \qquad \Longrightarrow \qquad \dot{\phi}\simeq -\frac{3H^2-V(\phi)}{3H^3f'(\phi)}\longrightarrow 0$
    \item $\underline{H\to \pm \infty, f'(\phi)=0} \qquad \Longrightarrow \qquad \dot{\phi}=-\sqrt{6H^2-2V(\phi)}\longrightarrow  -\infty$ \\
    De esta última ecuación, pues, vamos a considerar un $f'(\phi)$ tal que nunca se anule. Tomamos por ejemplo $f'(\phi)>0$ con un mínimo en $f'_m>0$. De esta forma obtenemos el siguiente caso 
    \item $\underline{H\to 0,f'(\phi)>0}\qquad \Longrightarrow \qquad \dot{\phi}\longrightarrow -\sqrt{-2V(\phi)}$ \\
    De esta última ecuación se deduce que si se admite como posible solución $H\to 0$, entonces para que la ecuación tenga sentido debe que $V(\phi)<0$ (es decir, vamos a pedir que el potencial sea negativo. Para una discusión más en detalle referirse a \citep{2011CQGra..28t4004L})
\end{itemize}





Tomando un potencial negativo, se sigue de la ecuación (\ref{V mayor}) que, en particular, se cumple
$$
\frac{d}{dt}[H(2+\dot{f}H)] \leq 0
$$
y por lo tanto
\begin{equation}\label{ready}
H(2+\dot{f}H)\leq C_0,\qquad t>0
\end{equation}
\begin{equation}\label{ready2}
H(2+\dot{f}H)\geq C_0,\qquad t<0
\end{equation}
con $C_0$ el valor inicial de la cantidad $H(2+\dot{f}H)$. Vamos a suponer que $(2+\dot{f}H)>0$ pero que, a su vez, el potencial se encuentra acotado de manera tal que la ecuación (\ref{V menor}) siga teniendo sentido. Teniendo esto en mente, si asumimos que la condición inicial cumple $C_{0}< 0$, entonces de la ecuación (\ref{ready}) se desprende que en este caso $H<0$. Luego, el universo se contrae en $t=0$ y, teniendo en cuenta (\ref{ready}) y la desigualdad $(2+\dot{f}H)>0$, se obtiene que
$$
H\leq\frac{C_0}{2+\dot{f}H}<0,\qquad t\geq0
$$
Es decir, si para $t=0$ el universo se está contrayendo, siempre se estará contrayendo en el futuro. Asimismo, para el pasado, la ecuación (\ref{ready2}) toma la forma
$$
H\geq \frac{C_0}{2+\dot{f}H},\qquad t\leq0
$$
Si ahora consideramos que la condición inicial es $C_{0}> 0$, entonces para el futuro se tiene que
$$
H\leq \frac{C_0}{2+\dot{f}H},\qquad t\geq0
$$
mientras que para el pasado, (\ref{ready2}) se convierte en
$$
H\geq \frac{C_0}{2+\dot{f}H}> 0,\qquad t\leq0
$$
Luego, si el universo se expande a $t=0$, siempre se estuvo expandiendo durante el pasado. Los resultados obtenidos muestran que estos casos no contemplan el caso de un universo gobernado por modelos cosmológicos cíclicos.


Volviendo a la ecuación (\ref{rama negativa}) y sus implicancias, podemos ver que como $\dot{\phi}$ es continua y no divergente y, en particular, en el infinito se anula, entonces se deduce que $\dot{\phi}$ debe estar acotada. Llamaremos $\dot{\phi}_1$, $\dot{\phi}_2$ a dichas cotas de manera tal que

$$
\dot{\phi}_2\leq\dot{\phi}\leq\dot{\phi}_1
$$

y mediante una simple integración se obtiene que por lo tanto

$$
\phi_0+\dot{\phi}_1t\leq\phi\leq\phi_0+\dot{\phi}_2t\qquad t\geq 0
$$
\begin{equation}\label{bondo}
\phi_0+\dot{\phi}_2 t\leq\phi\leq\phi_0+\dot{\phi}_1t\qquad t\leq 0
\end{equation}

Esto significa que el valor de $\phi$ se encuentra acotado entre dos funciones lineales del tiempo y, por ende, estará acotado para cualquier tiempo finito $t$.







%%%%%%%%%   Soluciones singulares y regulares

    
\section{Soluciones singulares y regulares} 


Volviendo a principios del capítulo, vemos que los escalares de curvatura construídos a partir de (\ref{wk}) son todos dependientes de $H$ y de $\dot{H}$, como por ejemplo

$$
R=6\dot{H} + 12H^2 \qquad R_{ij}R^{ij}=12 \left(\dot{H}+ 3H^2\right)^2,
$$


La discusión a continuación se basa en las posibles singularidades de $H$ y $\dot{H}$ y, consecuentemente, de $R$.



Para ello, resultan de vital importancia los límites estudiados en la sección anterior. En primer lugar, es necesario tener en cuenta que (\ref{H poli}) puede entenderse como una expresión algebraica con coeficientes determinados por $f'(\phi)$ y $f''(\phi)$, como así también del potencial $V(\phi)$. Si estas funciones existen, son continuas para algún valor finito de $\phi$, y nunca se anulan, se obtiene a partir de (\ref{H poli}) y (\ref{bondo}) que dichos coeficientes poseen un buen comportamiento para todo valor finito de $t$. De esta manera, podemos notar que esta consideración implica que estas funciones no tengan asíntotas verticales. Por otro lado, la consideración de que los coeficientes no se anulen es por simplicidad; de otra forma, el comportamiento de las raíces de un polinomio pueden resultar singulares cuando alguno de los coeficientes desaparece\footnote{Por ejemplo el caso de una función cuadrática cuyo coeficientes principal tiende a cero. Es fácil ver que, en este límite, una de las raíces diverge.}. Luego, teniendo en cuenta que los coeficientes son finitos y son funciones simétricas de las raíces, resulta que las raíces $H(\phi)$ de (\ref{H poli}) son finitas para todo tiempo finito $t$.

El siguiente paso es estudiar las posibles singularidades de $\dot{H}$. Para fijar ideas consideremos el siguiente ejemplo. Sea una teoría (ficticia) cuyo vacío se describe a partir de $H^2+\phi^2=1$. Resulta claro que, para el punto $(H,\phi)=(1,0)$, el valor de $H'(\phi)$ diverge. Si la dinámica es tal que el punto $(H,\phi)=(1,0)$ es alcanzado en un tiempo finito, luego $\dot{H}=H'(\phi)\dot{\phi}$ parece diverger debido a que  $H'(\phi)\to\infty$ en este punto. Sin embargo, también resulta que $\dot{\phi}$ tiende a cero en $(H, \phi)=(1, 0)$ dado que éste es un punto de retorno para $\phi$. De esta forma, surge una indeterminación del tipo $0.\infty$. Sin embargo, es posible tomar diferentes casos para los cuales los campos evolucionan alrededor del círculo con velocidad finita, como por ejemplo $H=\sin(t)$ y $\phi=\cos(t)$. En esos casos, la indeterminación $0.\infty$ arroja un resultado finito y $\dot{H}$ resulta finito en el punto de retorno.

Para el presente modelo, la situación resulta mas compleja puesto que la curva que antes describía el círculo, ahora es descripta por (\ref{H poli}). Aún así, es posible obtener varias conclusiones. La ecuación (\ref{H poli}) puede ser pensada como una ecuación polinómica de la forma  
$$
P_n(H^2)=\sum_{n}^{12} a_n H^{2n}=0,
$$

donde los coeficientes de esta ecuación son funciones de $\phi$ con buen comportamiento y, por lo tanto, un cambio infinitesimal $d\phi$ inducirá un cambio suave en los coeficientes $da_i$. De esta forma, un cambio $dH^2$ en alguna de las raíces del polinomio estará vinculado con las variaciones $da_i$ a través de la siguiente fórmula

\begin{equation}\label{deri}
    P'_n(H^2) dH^2=-\sum_{n}^{11} H^{2n} da_n
\end{equation}



Si esta igualdad no se satisface, $H^2+dH^2$ dejaría de ser una raíz. De la última relación, se deduce que


\begin{equation}\label{par}
    \frac{\partial H^2}{\partial a_i}=-\frac{H^{2n}}{P'_n(H^2)}
\end{equation}


De esta forma, la derivada (\ref{par}) tiene un buen comportamiento cuando $P_n(H^2)=0$ pero $P'_n(H^2)\neq 0$. Para una función polinómica, esta condición es la afirmación de que las raíces $H^2$ son simples. En otras palabras, si la evolución de $\phi$ es tal que dos raíces, $H^2_1$ y $H^2_2$, se fusiona en un tiempo finito $t_{0}$, luego, en este punto la derivada (\ref{par}) será divergente. Por otro lado, la derivada temporal de $H^2$ es


\begin{equation}\label{wu}
    \frac{dH^2}{dt}=\frac{\partial H^2}{\partial a_i}\frac{da_i}{d\phi}\dot{\phi}
\end{equation}



Dado que $\dot{\phi}$ nunca diverge, podemos afirmar que la derivada temporal (\ref{wu}) nunca diverge si el rango de valores de las funciones $f(\phi)$ (junto con sus derivadas primera y segunda) y del potencial $V(\phi)$ toman un rango de valores tales que las raíces no se fusionan. En ese caso, la curvatura se halla controlada y esto sugiere que el universo puede ser eterno.


Consideremos ahora la Ec. de Raychaudhuri, la cual como se ha visto antes es una herramienta fundamental a la hora de estudiar singularidades,

\begin{equation}\label{Raycha cosmo}
\frac{d\theta}{dt}=-R_{tt}-\frac{\theta^2}{3}
\end{equation}

donde $\theta$ es el escalar de expansión del universo en un tiempo dado, $t$, que en el caso homogéneo e isótropo se reduce a $\theta=3H$. Para el espacio-tiempo considerado se tiene que 

$$
R_{tt}=-3\left(\dot H + H^2\right)
$$

Reemplazando $R_{tt}$ en la ecuación (\ref{Raycha cosmo}) se obtiene la identidad trivial $0=0$. Sin embargo, una relación no trivial se obtiene utilizando la ecuación (\ref{ec mov gab 2}). Usando dicha ecuación, en el caso $\kappa=1$, y haciendo la misma redefinición que antes $8f(\phi)\to f(\phi)$ obtenemos que 

$$
R_{tt}=\frac{1}{2}(\rho_{eff}+3p_{eff})
$$
$$
=\dot{\phi}^2+\frac{3}{2}H^2 f''(\phi)\dot{\phi}^2+\frac{3}{2}H^2 f'(\phi)\ddot{\phi}+3 H\dot{H} f'(\phi)\dot{\phi}+\frac{3}{2} H^3 f'(\phi)\dot{\phi}-V(\phi)
$$

donde reemplazamos $\rho_{eff}$ y $p_{eff}$. De esta manera, la ecuación de Raychaudhuri (\ref{Raycha cosmo}) resulta

\begin{equation}
3(1+H\dot{f})\frac{dH}{dt}=-\frac{\dot{\phi}^2}{2}-\frac{3}{2}H^2 (\ddot{f}+H\dot{f})-3H^2+V(\phi)
\end{equation}

A su vez, mediante el uso de (\ref{1}) y (\ref{2}), la ecuación previa se puede reescribir como

\begin{equation}
\left(\frac{\dot{\phi}^2}{2}+V(\phi)\right)\frac{dH}{dt}=H^2\bigg(-\dot{\phi}^2+V(\phi)\bigg)-\frac{3}{2}H^4(\ddot{f}+H\dot{f})-3H^4
\end{equation}


En el caso en que $V(\phi)=0$, se puede ver fácilmente que se obtiene una singularidad: en dicho caso, si $H\neq0$, entonces si $\dot{\phi}\rightarrow0$ se obtiene que $\dot{H}\rightarrow\infty$. De esta forma, se obtiene un valor singular para $\dot{H}$ y consecuentemente para la curvatura $R$ (singularidad de curvatura). Sin embargo, vemos que si $V(\phi)\neq0$ puede suceder que dicha singularidad no suceda. Podríamos pensar, pues, que el potencial ``suaviza" la evolución del universo de manera tal de que nunca haya un cambio demasiado abrupto en $H$. 


