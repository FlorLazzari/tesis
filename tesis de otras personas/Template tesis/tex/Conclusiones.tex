\chapter{Conclusiones}\label{conclusiones}


Mediante el uso de herramientas de topolog�a y geometr�a diferencial, pero tambi�n sin dejar de lado el aspecto f�sico del tema, hemos repasado a lo largo de la tesis los Teoremas de Singularidad de Hawking-Penrose de Relatividad General. A su vez, mediante lo discutido en \citep{2011CQGra..28l5009F}, pudimos generalizar dichos teoremas bajo ciertas hip�tesis y aplicarlos al caso de un modelo inflacionario de Higgs. Cabe destacar que los modelos inflacionarios de Higgs no satisfacen las condiciones usuales de energ�a que se asume usualmente en estos teoremas, pero vimos que si el universo inicialmente se expande o contrae con un par�metro de expansi�n suficientemente grande, entonces la singularidad resulta inevitable. 

Por otro lado, en el cap�tulo \ref{GB} estudiamos ciertos modelos de gravedad de Gauss-Bonnet en donde notamos que ciertas hip�tesis sobre la materia en los Teoremas de Singularidad son violadas. Sin embargo, fue posible realizar un an�lisis sobre las singularidades y la evoluci�n de los campos de la teor�a. Vimos a su vez que cuando el potencial se halla ausente, en ciertos casos nuestros resultados sugieren la existencia de universos eternos si el acoplamiento al t�rmino de Gauss-Bonnet tiene un comportamiento controlado. A su vez, hemos estudiado el efecto que tiene el agregado de un t�rmino de potencial en la acci�n. Nuevamente, nuestros resultados sugieren la existencia de universos eternos. Estos resultados son interesantes ya que sugieren que las teor�as de Gauss-Bonnet permiten evitar la presencia de una Gran Explosi�n. Cabe destacar que no pudimos demostrar resultados similares en teor�as ordinarias.

Asimismo, pudimos presentar ciertos resultados universales sobre estas teor�as, independientes del modelo elegido. En el caso sin potencial, pudimos demostrar que cuando la energ�a cin�tica del campo escalar 
$\dot{\phi}$ tiende a cero, entonces se forma una singularidad. En casos con potencial no nulo, la singularidad es inevitable cuando $\dot{\phi}^2/2+V(\phi)$ se anula. Este resultado fue obtenido mediante un an�lisis meticuloso de la ecuaci�n de Raychaudhuri. Tambi�n pudimos demostrar que en ausencia de potencial, el universo puede contraerse o expandirse tan solo una vez. Lo mismo sucede para ciertas clases de potenciales, pero esto en cambio no es un resultado gen�rico.


La generalizaci�n de los Teoremas de Singularidad permiten aplicar el mismo a diversas teor�as en donde, pareciera a simple vista, no se cumplen las condiciones de energ�a. En esta tesis hemos dado una aplicaci�n a un modelo inflacionario de Higgs, pero cabe destacar que un razonamiento an�logo se puede aplicar a un campo escalar (real) acoplado a las ecuaciones de Einstein \citep{2011CQGra..28l5009F}, o hasta teor�as $f(R)$ \citep{2016JCAP...05..023A}. Siguiendo esta l�nea, creemos que ser�a interesante estudiar diversas teor�as de la literatura, viendo posibles aplicaciones de los teoremas. 

Por otro lado, siguiendo lo hecho con Gauss-Bonnet, creemos que ser�a interesante como proyecto a futuro analizar qu� sucede si se considera ahora - ya sea con potencial nulo o no - la curvatura espacial no nula y estudiar c�mo se modifican los resultados previos. Sin embargo, la experiencia con los c�lculos aqu� realizados sugieren que esto no es algo directo, y queda como un trabajo pendiente para una futura investigaci�n.

