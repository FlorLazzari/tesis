\chapter{Espacios topológicos}\label{esptopologicos}

En dicho apéndice se enuncian ciertas definiciones básicas en materia de topología, como así también se enuncian algunos teoremas (sin demostración) que serán usado a lo largo de la tesis .
%\section{Análisis espectral de datos equiespaciados}\label{subsec.fourier}


\begin{definition}{}
Un espacio topológico $(X,T)$ consiste en un conjunto $X$ junto a una colección $T$ de subconjuntos de $X$ tales que cumplen las siguientes tres propiedades:
\begin{enumerate}
\item La unión arbitraria de elementos en $T$, está en $T$ 
\item La intersección finita de elementos en $T$, está en $T$
\item $X$ y $\emptyset$ están en $T$
\end{enumerate}
A los elementos de $T$ se les llama los abiertos de $X$, y $T$ es la topología sobre $X$.
\end{definition}


\begin{definition}
Un espacio topológico, $(X,T)$, es de \textit{Hausdorff} si para cada par de puntos distintos $p,q\in X$, existen dos abiertos $U$ y $V$ tales que $p\in U$, $q\in V$ y  $U\cap V=\emptyset$
\end{definition}

\begin{definition}
Un mapa que lleva de una variedad $C^\infty$, $M$, a otra variedad ($C^\infty$) $N$, que es 1-1 y $C^\infty$, y su inversa también lo es, se dice un \textit{difeomorfismo} de $M$ a $N$, y se dice que $M$ y $N$ son \textit{difeomorfas}. 
\end{definition}

\begin{definition}
Un espacio topológico $(X,T)$ se dice \textit{conexo} si sus únicos abiertos y cerrados son $X$ y $\emptyset$.
\end{definition}

\begin{definition}
Sea $(X,T)$ un espacio topológico y $A\subseteq X$; una familia de abiertos $\{O_i\}$ tal que $A\subseteq\bigcup_{i}O_i$ se llama \textit{cubrimiento} de $A$. Una subfamilia de $\{O_i\}$ que también cubre $A$ se llama \textit{subcubrimiento} de $A$.
\end{definition}


\begin{definition}
Dado un espacio topológico $(X,T)$  y $A\subseteq X$, $A$ se dice \textit{compacto} si cada cubrimiento de $A$ tiene un subcubrimiento finito.
\end{definition}

\begin{theorem}
Sea $(X,T)$ de Hausdorff y $A\subseteq X$ compacto, entonces $A$ es cerrado.
\end{theorem}

\begin{theorem}
Sea $(X,T)$ compacto y $A\subseteq X$ cerrado, entonces $A$ es compacto.
\end{theorem}

Para el caso de un subconjunto de $\mathbb{R}$ se establece el teorema conocido de Heine-Borel:

\begin{theorem}
Un subconjunto $A$ de $\mathbb{R}^n$ es compacto sii es cerrado y acotado.
\end{theorem}

\begin{definition}
Una \textit{sucesión de Cauchy} es una sucesión infinita de puntos $x_n$ tales que para cualquier $\epsilon>0$, existe un número $N$ tal que $\rho(x_n,x_m)<\epsilon$ para $n,m>N$, donde $\rho(x,y)$ es la función distancia entre $x$ e $y$.
\end{definition}

\begin{definition}
Dado un espacio topológico $(X,T)$, un punto $p\in X$ es un \textit{punto límite} (o \textit{punto de acumulación}) de una sucesión $\{x_n\}$ si todo entorno abierto de $p$ contiene infinitos puntos de $\{x_n\}$.
\end{definition}

\begin{definition}
Un espacio topológico $(X,T)$ se dice \textit{primero contable} si para todo $p\in X$ existe una familia contable de abiertos $\{O_n\}$ tal que todo abierto que contenga a $p$, contiene también algún $O_n$.
\end{definition}

\begin{definition}
Un espacio topológico $(X,T)$ se dice \textit{segundo contable} si existe una familia contable de abiertos $\{O_n\}$  tal que todo abierto se puede escribir como unión de los $O_n$.
\end{definition}

Un importante teorema que relaciona la convergencia de series con la compacticidad es el de Bolzano-Weierstrass:

\begin{theorem}\label{B-W}
Sea $(X,T)$ un espacio topológico y $A\subset X$. Si $A$ es compacto entonces cada sucesión $\{x_n\}$ de puntos en $A$ tiene un punto límite dentro de A. Inversamente, si $(X,T)$ es segundo contable y cada sucesión en $A$ tiene un punto límite en $A$, entonces $A$ es compacto. En particular, si $(X,T)$ es segundo contable, $A$ es compacto sii cada sucesión en $A$ tiene una subsucesión convergente cuyo límite está dentro de $A$. 
\end{theorem}

\begin{definition}
Sea $(X,T)$ un espacio topológico y $\{O_i\}$ un cubrimiento de $X$. Un cubrimiento $\{V_j\}$ se dice una \textit{refinamiento} de $\{O_i\}$ si para cada $V_j$ existe un $O_i$ tal que $V_j\subseteq O_i$.
\end{definition}


\begin{definition}
Un refinamiento $\{V_j\}$ se dice \textit{localmente finito} si para cada $x\in X$ hay un entorno abierto $W$ tal que solo finitos $V_j$ satisfacen $W\cap V_j\neq \emptyset$.
\end{definition}

\begin{definition}
Un espacio topológico $(X,T)$ se dice \textit{paracompacto} si cada cubrimiento $\{O_i\}$ de $X$ tiene un refinamiento localmente finito $\{V_j\}$. 
\end{definition}


\begin{theorem}
Cualquier espacio topológico de Hausdorff que es localmente compacto (cada punto tiene un entorno abierto con clausura compacta) y que puede ser expresado como la unión de finitos conjuntos compactos, es paracompacto. 
\end{theorem}

Una consecuencia del teorema previo es que $\mathbb{R}^n,\mathbb{S}^n$ y sus productos son paracompactos. 

\begin{theorem}
Si $M$ es una variedad paracompacta entonces $M$ admite una métrica Riemanniana y es segundo contable.
\end{theorem}

\begin{definition}
Dado un cubrimiento localmente finito $\{O_i\}$ de $M$, una \textit{partición de la unidad} subordinada a $\{O_i\}$ es una familia de funciones suaves $\{f_i\}$ tales que (i) $supp(f_i)\subseteq O_i$, (ii) $0\leq f_i \leq 1$, (iii) $\sum_i f_i=1$.
\end{definition}


Una propiedad importante que cumplen las variedades paracompactas es la existencia de una partición de la unidad.
