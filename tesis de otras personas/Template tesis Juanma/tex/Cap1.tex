\chapter{Introducción}





La Teoría de la Relatividad General (RG), formulada por Einstein en 1915, es sin duda la teoría más exitosa a la hora de estudiar diversos aspectos del universo. Según la misma, el espacio y el tiempo conforman un único ente descripto matemáticamente por una variedad Lorentziana de dimensión 4, y su curvatura está determinada por la distribucion de energía y de materia del universo, relacionadas entre sí mediante las Ecuaciones de Einstein. Asumiendo hipotesis de alta simetría, varios autores han dado soluciones exactas para las mismas. En algunos casos, los espacio-tiempos obtenidos contenían ciertas patologias llamadas \textit{singularidades}. Por varios años se creyó que estas patologías se debían precisamente a la alta simetría asumida y que no se daría en modelos mas realistas del espacio-tiempo. Esto fue hasta que a fines de los 60', R. Penrose, S. Hawking y R. Geroch, mediante argumentos de topología y geometría diferencial, demostraron que cualquier espacio-tiempo que cumpla ciertas condiciones posee alguna singularidad.


A su vez, diversas observaciones sobre el universo (como por ejemplo su expansión acelerada) llevaron a diversos autores (Weyl, Starobinsky, Lovelock, etc) a proponer modelos alternativos de RG que logren explicar de manera adecuada lo observado. Una alternativa plausible es que la RG a escalas cosmologicas no describa de manera adecuada las interacciones gravitatorias. Por este motivo, es de interés el estudio de teorías de gravedad modificada, en las cuales se introducen términos de mayor orden en la acción de RG. En este contexto, es de interés saber si los Teoremas de Singularidad se pueden aplicar en una teoría de gravedad modificada.


Por otro lado, uno de los modelos más exitosos que logra explicar la aceleración del universo, es la propuesta de la existencia de materia y energía oscura. Sin embargo, es posible explicar la misma sin la presencia de materia y energía oscura, donde uno de los modelos más conocidos es el de \textit{Gauss-Bonnet} el cual hoy en día hay diversos campos donde adquiere relevancia: Renormalización en espacios curvos \citep{1982qfcs.book..340B,2017arXiv170601572F}, compactificación de la supercuerda heterótica \citep{1985NuPhB.262..593C,1982AnPhy.143..413F,1985PhRvD..32.2102S,1985PhRvL..55.1846S,1987NuPhB.291...41G}, gravedad cuántica \citep{2017arXiv170609315H,2017arXiv170700169K,2017arXiv170503161F}, etc.



El objetivo de esta tesis es, por un lado, presentar los resultados necesarios para estudiar los Teoremas de Singularidad de Hawking-Penrose y, luego, usar lo expuesto en \citep{2011CQGra..28l5009F} para dar una generalización de los mismos, aplicandolo luego al caso de un modelo inflacionario de Higgs. A su vez, otro de los objetivos de la tesis es el estudio de singularidades y la evolución del universo en una teoría de gravedad de Gauss-Bonnet, donde ahora no se satisfacen las hipótesis necesarias para poder usar los Teoremas de Singularidad vistos previamente. La tesis se organiza de la siguiente manera: en el capítulo \ref{geo dif}, se repasan los conceptos y resultados básicos de geometría diferencial, topología y Relatividad General. En el capítulo siguiente se brinda una explicación sobre la Estructura Causal del espacio-tiempo, dando definiciones y resultados que serán usado a lo largo de la tesis y que son de sumo interés para la misma. El capítulo \ref{capitulo Jacobi} es el más extenso de la tesis y en él se terminan de dar las herramientas y resultados necesarios para enunciar y demostrar luego los Teoremas de Singularidad. En dicho capítulo se debaten dos conceptos de sumo interés: puntos conjugados y Ecuación de Raychaudhuri, con las implicancias pertinentes en cada caso. En particular - al tratarse del capítulo más extenso - se trató de dejar amena la lectura y, por esa razón, se trató de ir entrelazando a lo largo del capítulo las definiciones y resultados con explicaciones que ayuden a entender (de una manera más intuitiva) lo desarrollado, como así también dejando de lado ciertas demostraciones que el lector podrá consultar en la bibliografía. En el capítulo \ref{sing} se presentan los Teoremas de Singularidades para geodésicas causales, haciendo la distinción entre temporales y nulas, dando luego la generalización de los mismos y aplicandolos a un modelo inflacionario de Higgs. Finalmente, en el capítulo \ref{GB}, se otorga una discusión sobre gravedad de Gauss-Bonnet en donde se estudia lo que sucede con la evolución y las soluciones singulares de la teoría cuando se tiene en cuenta un término de potencial no nulo en la acción.

