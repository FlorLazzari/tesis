% Energia Eólica

El potencial eólico de la Argentina es uno de los mayores del mundo. Actualmente la
participación de esta fuente de energı́a en el mercado eléctrico nacional es baja, pero se
espera que en los próximos años aumente considerablemente. Debido a esta expectativa, se
necesitan herramientas que faciliten la incorporación de mayores cantidades de energı́a de
esta fuente en el mercado eléctrico nacional.

La energía eólica es una forma de generación eléctrica renovable que ayuda a disminuir las emisiones de gases de efecto invernadero al reemplazar fuentes de energía a base de combustibles fósiles. Debido a la naturaleza variable del recurso eólico, la capacidad de pronosticar la velocidad del viento en el futuro es valiosa.
El viento es suficientemente estable y predecible a escala anual, aunque presenta variaciones significativas a escalas de tiempo menores. El pronóstico de producción de energía permite prever posibles variaciones en la generación, posibilitando una mejor gestión de la red eléctrica. Además, el modelado del flujo resulta de utilidad para la prospección e instalación de futuros proyectos.

El objetivo último de este proyecto es sentar las bases al desarrollo de modelos operativos
de pronóstico de potencia eólica y de modelado del flujo atmosférico en parques eólicos.

% Tesis de Cristian

Introducción general
1.1 Energía eólica: contexto global, regional y local

Hacia fines de 2015, la capacidad total instalada de turbinas eólicas en el mundo alcanzaba
valores record de 432.883 MW de potencia 1 continuando con un ritmo de crecimiento
exponencial, siendo 2015 el año que presentó el mayor crecimiento anual desde el año 2000
en adelante. Justamente desde el año 2000 hasta el presente, la potencia instalada a nivel
mundial aumentó aproximadamente 25 veces, mostrando un notorio impulso a la producción
de energía eléctrica por medio de aerogeneradores. La Figura 1.1, tomada el último informe
del Global Wind Energy Council (GWEC), presenta la evolución de la potencia mundial total
instalada a través de los años.


Este crecimiento exponencial observado en los últimos 15 años ha sido acompañado no sólo
por avances tecnológicos propios del diseño de los aerogeneradores sino también por políticas
económicas que han posibilitado su desarrollo. Asimismo, el comienzo hacia una nueva era de
concientización sobre la reducción de emisión de Gases de Efecto Invernadero (GEI) con el fin
de evitar el continuo aumento de la temperatura media global ha ido posicionando a la energía
eólica, como a otras energías alternativas y renovables, como un reemplazo viable a la
producción energética mediante la quema de combustibles fósiles.

Del total mundial de potencia instalada actualmente, 40.6% corresponde a Asia, 34.1% a
Europa, 20.5% a Norte América, y el restante 4.8% se reparte en el resto del mundo.
Con respecto a Latinoamérica, de los 12.220 MW de potencia instalados, 71.3% corresponde a
Brasil, 7.6% a Chile, 6.9% a Uruguay, y 2.3% a Argentina. El 11.9% restante se reparte en los
restantes países de Sudamérica, Centroamérica e islas del Caribe.
Las perspectivas de crecimiento de la potencia mundial instalada son positivas para el período
2016-2020 para cada uno de los continentes y subcontinentes considerados, estimando un
agregado total de 43.200 MW en Latinoamérica e islas del Caribe para finales del año 2020.
Con respecto al aprovechamiento local del recurso, el potencial eólico de la Argentina es uno
de los mayores del mundo, no sólo pensando en la región patagónica, sino también en otras
regiones del país, como el litoral marítimo bonaerense. Actualmente se encuentran instalados
y en ejecución alrededor de 279 MW de potencia en distintos parques eólicos distribuidos en
el territorio argentino, de un total de 31.815 MW de potencia total considerando todas las
fuentes de producción de energía de nuestro país instalada 2 indicando que menos del 1% de la
potencia proviene del recurso eólico. Asimismo la participación de la energía eólica es variable
para cada región, representando 13.9% del total de la potencia instalada en la región
patagónica, y 1.9% en el Noroeste Argentino, siendo despreciable para el resto de las regiones.
De los 279 MW de potencia eólica instalada en Argentina, 64% se distribuye en los siguientes
tres parques eólicos:
- Parque Eólico Rawson: Compuesto por 43 aerogeneradores Vestas V90 de 1.8 MW de
potencia nominal, instalado al sur de la localidad de Rawson, en la región noreste
patagónica, con una potencia total instalada de 77.4 MW. Actualmente es el parque
eólico más grande de Argentina.
- Parque Eólico Loma Blanca: Este parque eólico se encuentra instalado entre las
localidades de Puerto Madryn y Trelew, también en la región noreste patagónica. Está
compuesto por 17 aerogeneradores Alstom ECO100 de 3 MW de potencia, totalizando
51 MW de potencia instalada.
- Parque Eólico Arauco: Compuesto por 26 aerogeneradores IMPSA IWP-83 de 2.0 MW
de potencia, que totalizan 50.2 MW. Este parque eólico se encuentra instalado en el
Valle de Arauco, Provincia de La Rioja.
El tercio restante se reparte entre pequeños parques eólicos y ejecución de obras de
ampliación de parques eólicos existentes.
Sin embargo, y a pesar de este panorama actual en que la capacidad instalada dista mucho de
su potencial, nuevas políticas nacionales han sido propuestas a fin de promover el desarrollo
de energías renovables. El nuevo marco regulatorio argentino se basa en la Ley Nacional
27.191, relacionada al Régimen de Fomento Nacional para el Uso de Fuentes Renovables de
Energía Destinada a la Producción de Energía Eléctrica. Esta reciente Ley sancionada en el mes
