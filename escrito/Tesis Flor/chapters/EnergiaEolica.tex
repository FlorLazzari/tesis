\chapter{Energía Eólica}\label{EE}

\section{Contexto global y local}

% TESIS DE CRISTIAN


% aca se podrian agregar cosas de mi revista:

% MASTER THESIS
Aunque la energía eólica se ha utilizado durante siglos, la utilización a gran escala para la producción de energía eléctrica se ha implementado recientemente.
Las primeras turbinas eólicas utilizadas para este objetivo comenzaron a aparecer en la década de 1980. Normalmente los diámetros del rotor eran de aproximadamente 20 metros y podían producir entre 20 y 60 kW.
Hoy en día, los aerogeneradores con una potencia nominal de 4 a 6 MW son las unidades más vendidas y se están probando algunos con diámetros de rotor superiores a 100 metros.

La energía eólica es una forma de generación eléctrica renovable que ayuda a disminuir las emisiones de gases de efecto invernadero al reemplazar fuentes de energía a base de combustibles fósiles.

% Global Wind Energy Council
El Global Wind Energy Council (GWEC) publicó su Reporte Eólico Global, que muestra que la energía eólica es la tecnología limpia con el precio más competitivo en muchos de los mercados de todo el mundo. La aparición de sistemas híbridos eólicos-solares, una gestión más  sofisticada de la red y un almacenamiento cada vez más accesible comienzan a dar una idea de cómo se verá un sector de energía totalmente libre de combustibles fósiles.
La energía eólica instalada en el mundo creció un 9% en 2017, con lo que las instalaciones totales alcanzaron los 539 GW en todo el mundo.
La figura \ref{global_cumulative_wind_capacity.jpg}, tomada del último informe del GWEC, presenta la evolución de la potencia mundial total instalada a través de los años.

% TESIS DE CRISTIAN
Este crecimiento exponencial ha sido acompañado no sólo por avances tecnológicos propios del diseño de los aerogeneradores sino también por políticas económicas que han posibilitado su desarrollo. Asimismo, el comienzo hacia una nueva era de concientización sobre la reducción de emisión de gases de efecto invernadero con el fin de evitar el continuo aumento de la temperatura media global ha ido posicionando a la energía eólica, y otras energías alternativas y renovables, como un reemplazo viable a la producción energética mediante la quema de combustibles fósiles.

En el mismo informe del GWEC encontramos que los primeros productores mundiales son China, Estados Unidos, Alemania, India y España como se muestra en la figura \ref{top_10_cumulative_capacity}.
Del total mundial de potencia instalada actualmente, 40.6% corresponde a Asia, 34.1% a Europa, 20.5% a Norte América, y el restante 4.8% se reparte en el resto del mundo.

Con respecto al aprovechamiento local del recurso, el potencial eólico de la Argentina es uno de los mayores del mundo, particularmente en la región patagónica y el litoral marítimo bonaerense.
Actualmente menos del 1% de la energía eléctrica consumida en Argentina proviene del recurso eólico.

Sin embargo, y a pesar de este panorama actual en que la capacidad instalada dista mucho de su potencial, nuevas políticas nacionales han sido propuestas a fin de promover el desarrollo de energías renovables.
Hasta ahora las centrales eólicas no se tienen especialmente en consideración en el despacho del sistema eléctrico por ser muy pequeña su participación. No obstante, a medida que la potencia eólica instalada aumente, la energía suministrada por estos parques a las redes eléctricas deberá ser tenida en cuenta.


\section{Pronósticos de potencia eólica}

La capacidad de pronosticar el estado de la atmósfera en un futuro cercano (desde horas a días) y su conversión a potencia, resulta de gran interés tanto a los operadores del mercado eléctrico, como a los inversores cuyo objetivo es la construcción de un nuevo parque eólico y a los dueños de parques que ya están en funcionamiento.

% se entiende??:
A diferencia de otras fuentes de energía eléctrica, las cuales pueden controlar la producción dependiendo de la demanda eléctrica instantánea de los usuarios, la eólica no tiene la capacidad de controlar la inyección de energía hacia la red, ya que depende de las condiciones del tiempo meteorológico.
La predicción de potencia en parques eólicos es una herramienta muy útil ya que permite a los operadores del mercado eléctrico aprovechar los momentos de alta producción eólica y prever posibles variaciones en la generación.
En otras palabras, a los operadores el pronóstico de potencia les permite obtener información que los ayuda a la hora de tomar la decisión de qué fuentes de generación utilizar para cubrir la demanda eléctrica de la forma más económica y eficiente posible.

De esta forma, un buen pronóstico de potencia hace que el dueño de un parque pueda informar a los operadores del mercado eléctrico la cantidad aproximada de energía que podrá inyectar a la red.
% esto es asi?
Esta capacidad de anticipación posibilita una mejor gestión de la fuente de energía, lo que hace que el precio del recurso se reduzca y así se vuelva una fuente de energía eléctrica más competitiva en el mercado.

Por otro lado, a los inversores de futuros parques les sirve como herramienta para optimizar la distribución de los aerogeneradores
de forma que, teniendo en cuenta las condiciones meteorológicas y geográficas externas,
la potencia total generada por el parque sea máxima. En concluisión, el modelado del flujo resulta de utilidad para la prospección e instalación de futuros proyectos.

Finalmente, a los dueños de parques ya operativos, el pronóstico les permite planear y organizar los períodos de mantenimiento de forma inteligente tratando de reducir la pérdida de generación.
Debido a la naturaleza del viento, la fatiga de los aerogeneradores es altamente variable.
Por lo tanto, el pronóstico también les da a los dueños de los parques un panorama de la vida útil de sus turbinas y de esa forma se hace posible la estimación de la rentabilidad del proyecto.

Dependiendo de las necesidades operativas del pronóstico, existen distintas herramientas de variado costo computacional para modelar el flujo de aire atravesando un parque eólico.
%chequear con labo 1 si la palabra correcta es precisión en este caso:
Existe una relación de compromiso entre la exactitud del modelo y el costo computacional.
