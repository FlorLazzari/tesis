\chapter{Introducción}

Las colisiones protón-protón con grandes energías de centro de masa en el Gran Colisionador de Hadrones (LHC) ponen a prueba las leyes de la naturaleza a las distancias más pequeñas concebidas al momento. Los \textit{Jets}, ``chorros'' colimados de hadrones, son los objetos de estado final predominantes en las colisiones $pp$ de alta energía. Al ser la evidencia experimental de quarks y gluones energéticos producidos en estos procesos, los jets juegan un papel fundamental no sólo en mediciones de precisión de Modelo Estándar (SM), sino también en la búsqueda de nuevos fenómenos que empujen y motiven el entendimiento de la física más allá del SM. Como todos los objetos reconstruidos experimentalmente, los jets necesitan ser calibrados de manera de describir apropiadamente la escala de energías de partículas que los originan. El proceso de calibración de jets es un proceso largo y de varias etapas. En ATLAS, las dos más importantes son la MC+JES y las calibraciones \textit{in-situ}. La primera es la más significativa y se deriva enteramente a partir de simulaciones de MC, en las cuales se tiene acceso a cierta información que en el experimento no. La segunda provee de correcciones adicionales que dan cuenta de las diferencias que existen entre las simulaciones y los datos experimentales, que surgen, por un lado, por la naturaleza no perturbativa de muchos procesos de QCD, y por el otro, de la imposible tarea de simular fielmente tanto el detector como la interacción de las partículas con el mismo.     

Justamente porque el proceso de calibración de jets es largo y tedioso, solamente se tienen disponibles calibraciones completas y bien estudiadas para jets reconstruidos a partir de la escala EM y LC, con el algoritmo anti-k$_t$ y de radio 0.4. Sin embargo, los distintos análisis a realizarse con los datos de ATLAS podrían beneficiarse si pudieran utilizar Jets calibrados con otro tamaño. Esta tesis estudia un método de calibración llamado calibración R-scan. Con él, se pueden derivar correcciones de tipo \textit{in-situ} para jets que no han pasado por la cadena completa de calibración, inter-calibrando los mismos respecto de otros jets que sí. En este trabajo se derivan, dos calibraciones de tipo Rscan, junto con sus incertezas sistemáticas, para jets reconstruidos con el algoritmo anti-k$_t$ a partir de la escala LC con radio igual a 0.2 y radio igual a 0.6. 
Las calibraciones obtenidas en este trabajo se derivan usando datos de colisiones $pp$ a $\sqrt{s}=$13TeV recolectados por el detector ATLAS durante el año 2016.

Esta tesis se organiza de la siguiente manera. En el capítulo \ref{SM} se resumen las características principales del SM y QCD, y se introducen algunos conceptos relacionados con la interacción de las partículas con la materia, fundamentales para el entendimiento de los procesos de medición en ATLAS. En el capítulo \ref{TheExperiment} se describe al detector ATLAS indicando sus diferentes subsistemas. El capítulo \ref{jets} se centra en el concepto de \textit{jets}: qué son, cómo se reconstruyen y cómo se calibran. El método utilizado para derivar las calibraciones se desarrolla en el capítulo \ref{RscanCalib}, donde también se presentan los resultados obtenidos en este trabajo. En el capítulo \ref{Validation} se aplican las calibraciones derivadas para estudiar su validez, y en el capítulo \ref{Syst} se estudian las incertezas sistemáticas. En el capítulo \ref{dijets} se realiza una comparación entre la calibración obtenida en esta tesis a partir de eventos de $Z+jets$, con una calibración Rscan derivada a partir de eventos de dijets. Finalmente, en el capítulo \ref{Conclus} se presentan las conclusiones de esta tesis.
