

%% TABLAS

\begin{table}[ht]
    \centering
    \begin{tabular}{|l|l|l|l|l|} 
        \hline
        \multicolumn{2}{|c|}{input} & \multicolumn{2}{c|}{output} & \\
        \hline
        left & right & left & right & resulting movement\\ \hline
        0 & 0 & 1 & 1 & move forward\\ \hline
        0 & 1 & 1 & 0 & turn left\\ \hline
        1 & 0 & 0 & 1 & turn right\\ \hline
        1 & 1 & 0 & 0 & move backward\\ \hline
    \end{tabular}
    \caption{I/O truth table}
    \label{tab:truthtable}
\end{table}


%% ITEMS

The metrics to collect are :
\begin{itemize}
    \item The gaze position of the subject
    \item The environment directly faced by each subject (RGB and depth)
    \item The position of each subject in the environment
\end{itemize}



\begin{comment}

The
Z
+jet sample
The Z
(
→
μμ
)
+jet and Z
(
→
ee
)
+jet samples are collected using single-lepton HLTs with various
p
T
thresholds. Events are required to contain either two opposite-sign muons or two opposite-
sign electrons, fulfilling standard tight isolation and identification requirements [29, 30], with
|
η
|
<
2.3 and
p
T
>
20 GeV.  The dilepton
(
``
)
system is required to have
p
T,
ll
>
30 GeV and
|
m
ll
−
m
Z
|
<
20 GeV,  where
m
Z
is the mass of the Z boson.   The leading jet in the event is
required to have
|
η
|
<
1.3 and
p
T
>
12 GeV,  and to have a large angular separation in the
(
x
,
y
)
plane  with  respect  to  the  dilepton  system,
|
∆
φ
(
Z, 1st  jet
)
|
>
2.8.   Events  are  rejected
if there is any second jet with
p
T, 2nd jet
>
5 GeV not fulfilling the condition
p
T,2nd jet
/
p
T,Z
=
α
<
0.3.  The value of the cut on
|
∆
φ
(
Z, 1st jet
)
|
is such that it does not bias the distribution
of
α
for
α
<
0.3.  As will be explained in Section 6.3, the requirement on
α
is tightened from
the nominal value of 0.3 and the results are studied as a function of its value.   In the Z
(
→
ee
)
+jet analysis an additional requirement is enforced that no electron in the event lie within
∆
R
=
√
(
∆
φ
)
2
+ (
∆
η
)
2
=
0.5 of a jet.  The Z+jet selection is also used in Section 7.4, with the
additional requirement that the jet is tagged as coming from a b quark using the combined
secondary vertex tagger [31], with a typical tagging efficiency of 70% and a misidentification
probability for light-flavor jets of 1%

https://arxiv.org/pdf/1607.03663.pdf

\end{comment}

\begin{comment} PHOTOS y EVTGEN
https://twiki.cern.ch/twiki/bin/view/CMSPublic/SWGuidePhotosInterface
Photos is a precision MC program which simulates QED interference and the imission of photon radiaiton.
https://evtgen.hepforge.org/
EvtGen is a Monte Carlo event generator that simulates the decays of heavy flavour particles, primarily B and D mesons. It contains a range of decay models for intermediate and final states containing scalar, vector and tensor mesons or resonances, as well as leptons, photons and baryons. Decay amplitudes are used to generate each branch of a given full decay tree, taking into account angular and time-dependent correlations which allows for the simulation of CP-violating processes.
\end{comment}


PARA THEORETICAL OVERVIEW
http://hep.uchicago.edu/atlas/docs/Physics_Documentation/Minimum_Bias_Interactions_phys-pub-2005-7.pdf
definitions of hard interactions
