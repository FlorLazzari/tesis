\chapter{Aerogeneradores} 

%%%%%%%%%%%%%%%%%%%%%%%%%%%%%%%%%%%%%%%%%%%%%%%%%%%%%%%%%%%%%%%%%%%%%
% "Wind Energy Handbook - Wiley Sons"

Una turbina eólica es un dispositivo que permite extraer energía cinética del viento. Al extraer parte de la energía cinética, la velocidad del viento disminuye, pero solo se ve afectada la masa de aire que pasa a través del disco del rotor. Suponiendo que la masa de aire afectada permanece separada del aire que no pasa a través del disco del rotor (la cual no se desacelera), se puede dibujar una superficie límite que contenga a la masa de aire afectada y este límite se puede extender agua arriba y aguas abajo formando un largo tubo de corriente de sección transversal circular. No hay flujo de aire a través del límite, por lo que el caudal másico del aire que fluye a lo largo del tubo de corriente será el mismo para todas las posiciones a lo largo de las líneas de corriente dentro del tubo de corriente. Como el aire dentro del tubo de corriente se ralentiza, pero no se comprime, el área transversal del tubo de corriente debe expandirse para acomodar el aire con menor velocidad (Figura 3.1).

% todas las turbinas de viento, cualquiera sea su diseño, operan de esta manera

Aunque la energía cinética se extrae del flujo de aire, un cambio repentino en la velocidad no es posible ni deseable debido a las enormes aceleraciones y fuerzas que esto requeriría. Sin embargo, la energía de presión se puede extraer de forma escalonada. La presencia de la turbina hace que el aire que se aproxima, corriente arriba, gradualmente disminuya la velocidad de modo que cuando el aire llegue al disco del rotor su velocidad ya sea menor que la velocidad del viento libre. El tubo de corriente se expande como resultado del ralentizando y, debido a que aún no se ha hecho ningún trabajo sobre el aire %estático?? a que se  refiere?
, la presión aumenta para absorber la disminución de la energía cinética. A medida que el aire pasa a través del disco del rotor, hay una caída en la presión %estática?? estudiar este concepto 
tal que, al salir, el aire está por debajo del nivel de presión atmosférica. 

El aire continúa aguas abajo con velocidad y presión %estática 
reducidas: esta región del flujo se llama la estela. Eventualmente, muy abajo, la presión %estática 
en la estela debe regresar al nivel atmosférico para lograr el equilibrio. El aumento en la presión %estática 
es a expensas de la energía cinética y por lo tanto causa un ralentizado del viento aguas abajo. Por lo tanto, entre las condiciones muy lejos aguas arriba y de estela muy lejana no hay cambios en la presión %estática
, pero hay una reducción en la energía cinética.

% "Streeter, V.L., Fluid Mechanics, Example 3.5, McGraw–Hill Inc. (1966), New York."

Bernoulli's principle can be derived from the principle of conservation of energy. This states that, in a steady flow, the sum of all forms of energy in a fluid along a streamline is the same at all points on that streamline. This requires that the sum of kinetic energy, potential energy and internal energy remains constant.[2] Thus an increase in the speed of the fluid – implying an increase in both its dynamic pressure and kinetic energy – occurs with a simultaneous decrease in (the sum of) its static pressure, potential energy and internal energy. If the fluid is flowing out of a reservoir, the sum of all forms of energy is the same on all streamlines because in a reservoir the energy per unit volume (the sum of pressure and gravitational potential ρ g h) is the same everywhere.[4]

% "Wind Energy Handbook - Wiley Sons"

El concepto de Actuador Discal
El mecanismo descrito anteriormente explica la extracción de energía cinética, pero de ninguna manera explica lo que le sucede a esa energía; si bien puede ser utilizada como trabajo útil, también parte de dicha energía puede volver a vertirse en el viento como turbulencia y eventualmente disiparse en forma de calor. Podemos comenzar un análisis del comportamiento aerodinámico de turbinas eólicas sin ningún diseño específico de turbina, considerando solamente el proceso de extracción de energía. El dispositivo general que lleva a cabo esta tarea se llama un actuador discal (Figura 3.2). 
Aguas arriba del disco, el tubo de corriente tiene un área de sección transversal más pequeña que la del disco y aguas abajo un área más grande que el disco. La expansión del tubo de corriente se debe a que el caudal másico debe ser el mismo en todas partes. La masa de aire que pasa a través de una sección transversal dada del tubo de corriente en una unidad de longitud por unidad de tiempo es $\rho AU$, donde $\rho$ es la densidad del aire, $A$ es el área de la sección transversal y $U$ es la velocidad del flujo. El caudal másico debe ser el mismo en todas partes a lo largo del tubo de corriente, por lo tanto
$\rho A_{\infty} U_{\infty} = \rho A_d U_d = \rho A_W U_W$

El símbolo $\infty$ se refiere a las condiciones muy lejanas aguas arriba, $d$ se refiere a las condiciones en el disco y $W$ se refiere a las condiciones en la estela lejana. Es habitual considerar que el actuador discal induce una variación de velocidad que debe superponerse a la velocidad de flujo libre. La componente en la dirección del flujo de este flujo inducido en el disco viene dado por $-aU_{\infty}$, donde $a$ se llama factor de inducción de flujo axial, o el factor de entrada. En el disco, por lo tanto, la velocidad neta en la dirección del flujo
es
$U_d = U_{\infty} (1-a)$



Teoría de conservación de momento (conservación???)
El aire que pasa a través del disco sufre un cambio global en la velocidad $U_{\infty} - U_W$ y una tasa de cambio de momento igual al cambio global de velocidad multiplicado por la tasa de flujo másico:

Tasa de cambio de impulso  $= (U_{\infty} - U_W) \rho A_d U_d$

La fuerza que causa este cambio de impulso proviene enteramente de la diferencia de presión a un lado y otro del actuador discal, porque el tubo de corriente está completamente rodeado de aire a presión atmosférica, lo que daría una fuerza neta nula. Por lo tanto,

$(p_{d}^{+} - p_{d}^{-}) A_d = (U_{\infty} - U_W) \rho A_d U_{\infty} (1-a)$

Para obtener la diferencia de presión ($p_{d}^{+} - p_{d}^{-}$), la ecuación de Bernoulli se aplica por separado a las secciones aguas arriba y aguas abajo del tubo de corriente; las ecuaciones separadas son necesarias porque la energía total es diferente aguas arriba y aguas abajo.

La ecuación de Bernoulli establece que en condiciones estables, la energía total del flujo (energía cinética, energía de presión estática y energía potencial gravitatoria) permanece constante siempre que no se realice ningún trabajo sobre el fluido o por el fluido. Por lo tanto, para una unidad de volumen de aire,

$\frac{1}{2}\rho U^2 + p + \rho g h =$ constant

Por lo tanto, aguas arriba tenemos

$\frac{1}{2}\rho_{\infty} U_{\infty}^{2} + p_{\infty} g h_{\infty} = \frac{1}{2} \rho_{d} U_{d}^{2} + p_{d}^{+} + \rho_{d} g h_{d}$

Suponiendo que el flujo es incompresible ($\rho_{\infty} = \rho_{d}$) y horizontal ($h_{\infty} = h_{d}$), entonces

$\frac{1}{2}\rho U_{W}^2 + p_{\infty} = \frac{1}{2} \rho U_{d}^{2} + p_{d}^{+}$

Del mismo modo, aguas abajo,

$\frac{1}{2}\rho U_{W}^2 + p_{\infty} = \frac{1}{2} \rho U_{d}^{2} + p_{d}^{-}$

Restando estas ecuaciones obtenemos

$(p_{d}^{+} - p_{d}^{-}) = \frac{1}{2}\rho (U_{\infty}^{2} - U_{W}^{2})$

La ecuación (3.4) luego da

$\frac{1}{2}\rho(U_{\infty}^2 - U_{W}^{2}) A_d = (U_{\infty} - U_{W}) \rho A_{d} U_{\infty}(1-a)$

y entonces

$U_{W} = (1-2a)U_{\infty}$

Es decir, mitad de la pérdida de velocidad axial en el tubo de corriente tiene lugar aguas arriba del actuador discal y la otra mitad aguas abajo.


%Power coefficient
%The force on the air becomes, from Equation (3.4)
%
%sgh
%
%As this force is concentrated at the actuator disc the rate of work done by the force
%is FU d and hence the power extraction from the air is given by
%Power 1⁄4
%lkjkl
%
%A power coefficient is then defined as
%
%;llm;
%
%where the denominator represents the power available in the air, in the absence of
%the actuator disc. Therefore,
%
%lkmlkm
%
%
%The Betz limit
%The maximum value of C P occurs when
%
%agdhdh
%
%which gives a value of a 1⁄4 13 .
%Hence,
%
%dhsdh
%
%The maximum achievable value of the power coefficient is known as the Betz
%limit after Albert Betz the German aerodynamicist (119) and, to date, no wind
%turbine has been designed which is capable of exceeding this limit. The limit is
%caused not by any deficiency in design, for, as yet, we have no design, but because
%the stream-tube has to expand upstream of the actuator disc and so the cross section
%of the tube where the air is at the full, free-stream velocity is smaller than the area
%of the disc.
%C P could, perhaps, more fairly be defined as
%
%dgadfg
%
%but this not the accepted definition of C P .
%
%The thrust coefficient
%The force on the actuator disc caused by the pressure drop, given by Equation (3.9),
%can also be non- dimensionalized to give a Coefficient of Thrust C T
%
%dfgfg
%A problem arises for values of a > 12 because the wake velocity, given by
%(1  2a)U 1 , becomes zero, or even negative; in these conditions the momentum
%theory, as described, no longer applies and an empirical modification has to be
%made (Section 3.5).
%The variation of power coefficient and thrust coefficient with a is shown in Fig-
%ure 3.3.
%
%%%%%%%%%%%%%%%%%%%%%%%%%%%%%%%%%%%%%%%%%%%%%%%%%%%%%%%%%%%%%%%%%%%%%
%% "2013_Wake_Modeling_Using_OpenFOAM"
%
%Before using the uniformly loaded actuator disk library
%available in the OpenFOAM package, its codes need to be
%checked and verified for the case of simulating a wind farm.
%Eq. (4) and (5) are used within the library to calculate the
%thrust generated by the actuator disk:
%
%$T = 2 \rho A U_{up}^{2} a (1-a)$
%
%$a = 1-\frac{C_P}{C_T}$
%
%where $U_{up}$ represents the upstream velocity. User has to
%give the upstream point coordinates for which the program
%calculates the velocity and substitutes it in Eq. (4). Results
%of this approach are significantly sensitive to the upstream
%point. For a case where only one wind turbine is situated in
%the domain or several turbines which are not in the wake of
%the others, the upstream point can be somewhere in front
%and far enough from the disk, so that its flow velocity is not	
%disturbed by the disk. The problem arises for the wind
%directions at which some wind turbines are found in a line
%parallel to the wind flow. At this condition, the upstream
%points of the wind turbine are in the downstream region of
%another one. For such a case, which happens frequently in
%a wind farm, this approach is not reasonable anymore. A
%possibility to avoid this problem is to transform Eq. (4).
%From actuator disk theory:
%
%$U_{up} = \frac{U_{disk}}{1-a}$
%
%where $U_{disk}$ is the wind velocity at the disk. Replacing $U_{up}$
%by $U_{disk}$ in Eq. (4) results in:
%
%$T = 2\rho A U_{disk}^2 \frac{a}{1-a}$
%
%Besides, Eq. (5) is not a proper way to calculate the axial
%induction factor a, because the thrust coefficient C T is a
%function of the velocity, but the velocity is not already
%known and is to be calculated by the simulation itself.
%Alternatively, by referring to a C T curve for REpower 5M and
%AREVA M5000, value of the axial induction factor, as a 
%function of U disk , is calculated and entered into the codes. This
%enables the modified library to calculate the thrust for dif­
%ferent wind speeds.­
%

% Wind Turbine - Fundamentals, Technologies, Application

The primary component of a wind turbine is the energy converter which transforms the kinetic energy contained in the moving air, into mechanical energy. The extraction of mechanical energy from a stream of moving air with the help of a disk-shaped, rotating wind energy converter follows its own basic rules. 
The credit for having recognised this principle is owed to Albert Betz. Between 1922 and 1925, Betz published writings in which he was able to show that, by applying elementary physical laws, the mechanical energy extractable from an air stream passing through a given cross-sectional area is restricted to a certain fixed proportion of the energy or power contained in the air stream. Moreover, he found that optimal power extraction could only be realised at a certain ratio between the flow velocity of air in front of the energy converter and the flow velocity behind the converter.
Although Betz’s "momentum theory”, which assumes an energy converter working without losses in a frictionless airflow, contains simplifications, its results are quite usable for performing rough calculations in practical engineering. But its true significance is founded in the fact that it provides a common physical basis for the understanding and operation of wind energy converters of various designs. For this reason, I will study a summarised mathematical derivation of the elementary "momentum theory” by Betz. 

Betz’s Elementary Momentum Theory
The kinetic energy of an air mass $m$ moving at a velocity $v$ can be expressed as:
$E = \frac{1}{2}mv^2$
Considering a certain cross-sectional area $A$, through which the air passes at velocity $v$, the volume $\dot{V̇}$ flowing through during a certain time unit, the so-called volume flow, is:
$\dot{V} = vA$
and the mass flow with the air density $\rho$ is:
$\dot{m} = \rho v A $

The equations expressing the kinetic energy of the moving air and the mass flow yield the amount of energy passing through cross-section A per unit time. This energy is physically identical to the power P:
$\frac{1}{2}\rho {v}^3 A $

The question is how much mechanical energy can be extracted from the free-stream airflow by an energy converter. As mechanical energy can only be extracted at the cost of the kinetic energy contained in the wind stream, this means that, with an unchanged mass flow, the flow velocity behind the wind energy converter must decrease. Reduced velocity, however, means at the same time a widening of the cross-section, as the same mass flow must pass through it. It is thus necessary to consider the conditions in front of and behind the converter (Fig. 4.1).
Here, $v_{\inty}$ is the undelayed free-stream velocity, the wind velocity, before it reaches the converter, whereas $v_{W}$ is the flow velocity behind the converter. The mechanical energy which the disk-shaped converter extracts from the airflow corresponds to the power difference of the air stream before and after the converter:
$P = \rho A_{\infty} {v_{\infty}}^3 − \rho A_W {v_W}^3 = \rho A_{\infty} {v_{\infty}}^3 − A_W {v_W}^3$

Maintaining the mass flow (continuity equation) requires that:
$\rho v_{\infty} A_{\infty} = \rho v_{W} A_{W}$
Thus,
$P = \frac{1}{2}$




